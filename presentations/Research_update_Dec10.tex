\documentclass{beamer}
\usetheme[white]{Wisconsin}
\usepackage{longtable}
\usepackage{listings}
\usepackage{color}
%% The amssymb package provides various useful mathematical symbols
\usepackage{amssymb}
%% The amsthm package provides extended theorem environments
\usepackage{amsthm} \usepackage{amsmath}
\usepackage[mathcal]{euscript} \usepackage{color}
\usepackage{textcomp}
\usepackage{algorithm,algorithmic}
\usepackage[retainorgcmds]{IEEEtrantools}
\usepackage[absolute,overlay]{textpos}
  \setlength{\TPHorizModule}{1mm}
  \setlength{\TPVertModule}{1mm}
\definecolor{listinggray}{gray}{0.9}
\definecolor{lbcolor}{rgb}{0.9,0.9,0.9}
\lstset{
  backgroundcolor=\color{lbcolor},
  tabsize=4,
  rulecolor=,
  language=c++,
  basicstyle=\scriptsize,
  upquote=true,
  aboveskip={1.5\baselineskip},
  columns=fixed,
  showstringspaces=false,
  extendedchars=true,
  breaklines=true,
  prebreak =
  \raisebox{0ex}[0ex][0ex]{\ensuremath{\hookleftarrow}},
  frame=single,
  showtabs=false,
  showspaces=false,
  showstringspaces=false,
  identifierstyle=\ttfamily,
  keywordstyle=\color[rgb]{0,0,1},
  commentstyle=\color[rgb]{0.133,0.545,0.133},
  stringstyle=\color[rgb]{0.627,0.126,0.941},
}

%% colors
\setbeamercolor{boxheadcolor}{fg=white,bg=UWRed}
\setbeamercolor{boxbodycolor}{fg=black,bg=white}

%%----------------------------------------------------------------------------%%
\author{Luke J. Kersting
    \\ NEEP
    \\ University of Wisconsin - Madison
    \\ FRENSIE Meeting
}

\date{\today}
\title{Electron Hard Scattering in FRENSIE}
\begin{document}
\maketitle

%%----------------------------------------------------------------------------%%
\begin{frame}{Introduction}

  \begin{itemize}
    \item Simulation of hard electron transport events
      \medskip
    \item Atomic excitation events
      \medskip
    \item Hard elastic scattering events
      \medskip
    \item Electroionization events
      \medskip
    \item Bremmstrahlung events
  \end{itemize}

\end{frame}

%%----------------------------------------------------------------------------%%
\begin{frame}{Electron Transport in Monte Carlo Codes}

  \begin{block}{MCNP}
    \begin{itemize}
      \item Historically has only used a condensed-history approched with mutiple scattering techniques.
      \item MCNP6 implemented a single-event method for energies below 1 keV, 
              were the condensed-history method no longer holds.
    \end{itemize}
  \end{block}
    
  \begin{block}{Penelope}
    \begin{itemize}
      \item Implenments a mixed method that simulates soft (condensed-history) 
              events below a cutoff energy/angle and hard (single-events) above.
    \end{itemize}    
  \end{block}

  \begin{block}{Frensie}
    \begin{itemize}
      \item Ultimately working towards implementing a mixed method while using 
              the cross-sectional data from MCNP6 for hard events 
    \end{itemize}    
  \end{block}
    

\end{frame}

%%----------------------------------------------------------------------------%%
\begin{frame}{Atomic Excitation}

  \begin{itemize}
    \item There is no angular deflection.
    \item There are no secondary particles.
    \item Energy dependent electron energy loss are tabulated in ACE tables.
    \item No sampling is required for this process.   
  \end{itemize}

\end{frame}

%%----------------------------------------------------------------------------%%
\begin{frame}{Hard Elastic Scattering}
  
  \begin{itemize}
    \item There is no energy loss.
    \item There are no secondary particles.
    \item ACE tables provide histogram CDF of the outgoing angle cosine, \textmu, 
            for $14-16$ energy groups.
    \item for $\mu > 0.999999$ an analytical function, $f(\mu)$, is used to 
            compute the scattering angle
  \end{itemize}

  \begin{equation*}
    f(\mu) = \frac{A}{(\eta + 1 - \mu)^2}
  \end{equation*}

  \begin{equation*}
    \eta(E,Z) = \frac{1}{4}\left(\frac{\alpha mc}{0.885p}\right)^2 Z^{2/3}[1.13+3.76(\alpha Z/\beta)^2]
  \end{equation*}


\end{frame}

%%----------------------------------------------------------------------------%%
\begin{frame}{Electroionization}
  
  \begin{itemize}
    \item The subshell is directly sampled.
    \item A knock-on electron is ejected.
    \item ACE tables provide CDF of the knock-on energy, $E_{knock}$, based on the incident electron energy.
    \item The incident electron energy is reduced by the $E_{knock} + E_{binding}$.
    \item Conservation of momentum is used to find the scattering and ejection angles (need to double check).
    \item Vacancy will be filled using atomic relaxation data (still need to implement). 
  \end{itemize}

\end{frame}

%%----------------------------------------------------------------------------%%
\begin{frame}{Electroionization Scattering Angle}
  \begin{itemize}
    \item Conservation of Momentum
  \end{itemize}

  \begin{equation*}
    (p_{knock}c + p_{a}c)^2 = (pc)^2 + (p'c)^2 - 2pp'cos(\theta)
  \end{equation*}

  \begin{equation*}
    (pc)^2 + (p'c)^2 - 2pp'cos(\theta) = (p_{knock}c)^2 =
              E_{knock}^2 - (m_{knock}c^2)^2
  \end{equation*}

  \begin{itemize}
    \item Conservation of Energy
  \end{itemize}

  \begin{equation*}
    T + m_ec^2 + M_ac^2 =T' + m_ec^2 + T_a + M'_ac^2 + T_knock + m_ec^2
  \end{equation*}

  \begin{equation*}
    T = T' + T_{knock} + E_{binding} = T' + W 
  \end{equation*}


\end{frame}

%%----------------------------------------------------------------------------%%
\begin{frame}{Bremmstrahlung}

  \begin{itemize}
    \item A photon is ejected.
    \item ACE tables provide CDF of the photon energy, $E_{\gamma}$, based on the incident electron energy.
    \item The incident electron energy is reduced by the $E_{\gamma}$.
    \item The electron direction is assumed to be essentially unchanged.
    \item For $1 keV <  E_{\gamma} < 1 GeV$ the direction of the photon is 
           sampled using a table based scheme from the condensed-history method (still need to implement). 
    \item For all other energies a analytical function, $p(\mu)$, is used (not 
           really appropriate for low energies).
  \end{itemize}

  \begin{equation*}
    p(\mu)d\mu = \frac{(1-\beta^2)}{2(1-\beta\mu)^2}d\mu
  \end{equation*}


\end{frame}

%%----------------------------------------------------------------------------%%
\begin{frame}{Conclusion}

  \begin{itemize}
    \item Once finished Frensie should be capable of forward electron transport below 1 keV.
    \item The ACE tables has cross-sectional data for energies between 10 eV and 
            100 GeV, but bin sizes may be too course for used above 1 keV.
    \item Next step is to implement condensed-history method.
  \end{itemize}

\end{frame}

\end{document}
