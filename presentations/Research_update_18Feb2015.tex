\documentclass{beamer}
\usetheme[white]{Wisconsin}
\usepackage{longtable}
\usepackage{listings}
\usepackage{color}
%% The amssymb package provides various useful mathematical symbols
\usepackage{amssymb}
%% The amsthm package provides extended theorem environments
\usepackage{amsthm} \usepackage{amsmath}
%% \usepackage{eqnarray}
\usepackage[mathcal]{euscript} \usepackage{color}
\usepackage{textcomp}
\usepackage{algorithm,algorithmic}
\usepackage[retainorgcmds]{IEEEtrantools}
\usepackage[absolute,overlay]{textpos}
  \setlength{\TPHorizModule}{1mm}
  \setlength{\TPVertModule}{1mm}
\definecolor{listinggray}{gray}{0.9}
\definecolor{lbcolor}{rgb}{0.9,0.9,0.9}
\lstset{
  backgroundcolor=\color{lbcolor},
  tabsize=4,
  rulecolor=,
  language=c++,
  basicstyle=\scriptsize,
  upquote=true,
  aboveskip={1.5\baselineskip},
  columns=fixed,
  showstringspaces=false,
  extendedchars=true,
  breaklines=true,
  prebreak =
  \raisebox{0ex}[0ex][0ex]{\ensuremath{\hookleftarrow}},
  frame=single,
  showtabs=false,
  showspaces=false,
  showstringspaces=false,
  identifierstyle=\ttfamily,
  keywordstyle=\color[rgb]{0,0,1},
  commentstyle=\color[rgb]{0.133,0.545,0.133},
  stringstyle=\color[rgb]{0.627,0.126,0.941},
}

%% colors
\setbeamercolor{boxheadcolor}{fg=white,bg=UWRed}
\setbeamercolor{boxbodycolor}{fg=black,bg=white}

%%----------------------------------------------------------------------------%%
\author{Luke J. Kersting
    \\ NEEP
    \\ University of Wisconsin - Madison
    \\ FRENSIE Meeting
}

\date{\today}
\title{Update: Electron Mode in FRENSIE}
\begin{document}
\maketitle

%%----------------------------------------------------------------------------%%
\begin{frame}{Outline}

  \begin{block}{Electron Mode}
    \begin{itemize}
      \item Single Scattering Events from 100 GeV to 10 eV
      \item Elastic, Bremsstrahlung, Electroionization, Atomic Excitation 
      \item Secondary particles created, but photons not tracked
      \item Atomic relaxation implemented
    \end{itemize}
  \end{block}
    
  \begin{block}{Adjoint Papers}
    \begin{itemize}
      \item Hybrid Multigroup/Continuous-Energy Monte Carlo using Boltzmann-Fokker-Planck Equation
      \item Discrete Scattering Angles and Discrete Energy Losses
    \end{itemize}    
  \end{block}  

\end{frame}

%%----------------------------------------------------------------------------%%
\begin{frame}{Electron Mode}

  \begin{block}{Capabilities}
    \begin{itemize}
      \item Single Scattering Events from 100 GeV to 10 eV
      \item Elastic, Bremsstrahlung, Electroionization, Atomic Excitation 
      \item Secondary particles created, but photons not tracked
      \item Atomic relaxation implemented
    \end{itemize}
  \end{block}
    
  \begin{block}{Problems}
    \begin{itemize}
      \item Absorption at low energies
      \item Negative energy from Electroionization
    \end{itemize}    
  \end{block}  

\end{frame}

%%----------------------------------------------------------------------------%%
\begin{frame}{Absorption at low energies}

  \begin{itemize}
    \item At energies near the cutoff (10 eV) the reaction cross section is dominated by elastic scattering (by order $10^7$ for H)
    \item It is unlikely the electron will scatter below the cutoff energy 
    \item A temporary fix is to raise the cutoff energy (to 15eV for H) to prevent indefinite elastic scattering
    \item No mention of this issue in MCNP or Penelope 
  \end{itemize}

\end{frame}

%%----------------------------------------------------------------------------%%
\begin{frame}{Negative energy from Electroionization}
  
  \begin{itemize}
    \item ACE tables provide CDF of the knock-on energy, $E_{knock}$, based on the incident electron energy.
    \item When the incident electron energy is between two tables a weighted random variable is used to chose the appropriate table
    \item This can result in a $E_{knock}$ that is larger than physically possible
     \item In this case the energy of incident electron is reduce to $1E{-15}$
    \item MCNP avoids this by interpolation between tables, which is more computationally expensive
  \end{itemize}


\end{frame}

%%----------------------------------------------------------------------------%%
\begin{frame}{Next Step}
  
  \begin{block}{Testing}
    \begin{itemize}
      \item Run tests in MCNP and FRENSIE for comparison
      \item Start with Hydrogen spheres
    \end{itemize}
  \end{block}
  
    \begin{block}{Possible Further Work}
    \begin{itemize}
      \item Create testing mode were no secondary particles are created
      \item Implement other options for the bremsstrahlung photon ejection angle
    \end{itemize}
  \end{block}

\end{frame}

%%----------------------------------------------------------------------------%%
\begin{frame}{Hybrid Multigroup/Continuous-Energy Monte Carlo}
 
  \begin{block}{ Boltzmann-Fokker-Planck Equation}
    \begin{eqnarray*}
      \Omega \cdot \nabla \psi + \sigma_t\psi &=& \int_{0}^{\infty}\int_{0}^{2\pi}\int_{-1}^{+1} \sigma_s(E'\rightarrow E,\mu_0)
       \times\psi(\mu',\phi',E')d\mu'd\phi'dE' \\
       & &+ \frac{\alpha}{2}\Big\{ \frac{\partial}{\partial \mu} \Big[(1-\mu^2)\frac{\partial \psi}{\partial \mu}\Big]+ \frac{1}{1-\mu^2}\frac{\partial^2\psi}{\partial\phi^2}\Big\} + \frac{\partial}{\partial E} [S\psi]+Q
    \end{eqnarray*}
   \end{block}
    
  \begin{block}{Modify Angular Fokker-Planck Operator}
    \begin{itemize}
      \item Replace $~~~~F_{\alpha}\psi=\frac{\alpha}{2}\Big\{ \frac{\partial}{\partial \mu} 
        \Big[(1-\mu^2)\frac{\partial \psi}{\partial \mu}\Big]+ \frac{1}{1-\mu^2}\frac{\partial^2\psi}{\partial\phi^2}\Big\}$
      
      \item With $~~~~B_{\alpha}\psi=\int_{0}^{2\pi}\int_{-1}^{+1} \sigma_a(E,\mu_0)\psi(\mu',\phi',E)d\mu'd\phi' - \sigma_a\psi$
        
      \item Where $~~B_{\alpha}\psi = F_{\alpha}\psi~$ in the limit as $~\mu_s \rightarrow 1$
      \item Assume error from high-order flux moments are small with highly forward-peaked scattering (holds for condensed history where the scale lengths are large compared to mfp)
    \end{itemize}
  \end{block}

\end{frame}

%%----------------------------------------------------------------------------%%
\begin{frame}{Hybrid Multigroup/Continuous-Energy Monte Carlo}
 
  \begin{block}{ Hybrid Multigroup/Continuous-Energy Approximation}
  
   Replace the Stopping Power, $S$ with $\tilde S$ \\
      $$\tilde{S}(E)=\sum_{g=1}^{N}S_gB_g(E)$$
   Where \\ $~~B_g(E)=1.0$,~~ for $~E \in (E_{g+1/2},E_{g-1/2})$ ~~~else $B_g(E)=0.0$ \\ 
   $~~S_g$ is the weighted group average of $S(E)$ using Radau quadratures

   \end{block}
    
  \begin{block}{Legendre Cross-Section Expansion}
    \begin{itemize}
      \item Replace S with \~S  
      \item Where $~~B_{\alpha}\psi = F_{\alpha}\psi~$ in the limit as $~\mu_s \rightarrow 1$
      \item Assume error from high-order flux moments are small with highly forward-peaked scattering (holds for condensed history where the scale lengths are large compared to mfp)
    \end{itemize}
  \end{block}

\end{frame}

\end{document}
