% FRENSIE tree flowchart
% Luke Kersting 10/2/2014
% \documentclass{article}
\documentclass[landscape]{letter}
\usepackage{tikz}
\usepackage{lscape}
\usepackage{tikz-qtree}

\begin{document}
\usetikzlibrary{shapes.geometric, arrows, trees}
\tikzstyle{main} = [rectangle, rounded corners, minimum width=4cm, 
minimum height=1cm, text centered, draw=black]

\tikzstyle{dependency} = [trapezium, trapezium left angle=60, trapezium right angle=120, 
minimum width=3cm, minimum height=0.5cm, text centered, draw=black, fill=gray!30]

\tikzstyle{photon} = [rectangle, minimum width=2.2cm, minimum height=0.75cm, 
text centered, draw=black, fill=blue!30]

\tikzstyle{electron} = [rectangle, minimum width=2.2cm, minimum height=0.75cm, 
text centered, draw=black, fill=green!30]

\tikzstyle{neutron} = [rectangle, minimum width=2.2cm, minimum height=0.75cm, 
text centered, draw=black, fill=red!30]

\tikzstyle{sub} = [rectangle, minimum width=2.2cm, minimum height=0.75cm, 
text centered, draw=black]

\tikzstyle{arrow} = [thick,->,>=stealth]
\tikzstyle{line} = [thick]

% Collision Tree
\begin{tikzpicture}[remember picture, overlay, node distance=1.5cm, 
grow'=down, level distance=1.5cm,sibling distance=3.5cm]

\tikzset{edge from parent/.style= 
{thick, draw, edge from parent fork down}
}

% FRENSIE tree
	\node [main, xshift=6.05cm] (0a) {\Large FRENSIE}
 		% First Branch
        child { node[electron] (1a) {\Large scripts} }
		child { node[electron] (1b) {\Large patches} }
		child {	node[electron] (1c) {\Large \bf packages} 
			% Second Branch
			child { node[photon] (2a) {\large utility} }
			child { node[photon] (2b) {\large transmutation} }
			child { node[photon] (2c) {\large \bf monte carlo}
				%Third Branch 
				child { node[neutron] (3a) {\large source} }
				child { node[neutron] (3b) {\large manager} }
				child { node[neutron] (3c) {\large estimator} }
				child { node[neutron] (3d) {\large doc} }
				child { node[neutron] (3e) {\large \bf core} }
				child { node[neutron] (3f) {\large collision}  
					%Fourth Branch
					child { node[neutron, fill=orange!25] (4a) {\large core} }
					child { node[neutron, fill=orange!25] (4b) {\large interface} }
					child { node[neutron, fill=orange!25] (4c) {\large \bf native} 
						%Fifth Branch
						child { node[neutron, fill=yellow!40] (5a) {\large test} }
						child { node[neutron, fill=yellow!40] (5c) {\large \bf src} }
					} 
				}
			}
			child { node[photon] (2d) {\large geometry} }
			child { node[photon] (2e) {\large data} }
		}
		child {	node[electron] (1d) {\Large cli} }
		child {	node[electron] (1e) {\Large cmake} }
		child {	node[electron] (1f) {\Large data} }
		child {	node[electron] (1g) {\Large doc} };

\end{tikzpicture}
\newpage
%Collision programs
\begin{tikzpicture}[remember picture, overlay, node distance=1.5cm,level distance=3.5cm,sibling distance=0.5cm]
\tikzset{edge from parent/.style= {level distance=1cm, thick, draw, edge from parent fork right}, grow'=right, ->, parent anchor=east }


\Tree
    [.\node [neutron, xshift=-5cm] (core) {\Large core};
        [.\node [dependency, xshift=-4.7cm, yshift=-7cm] (core2) {Particle Bank}; ]
        [.\node [dependency, xshift=-4.7cm, yshift=2cm] (core1) {Photon State}; 
			% Photoatomic Reaction
			[.\node [sub, xshift=-3cm] (Patomic) {\Large Photoatomic reaction}; 
				[.\node [sub, xshift=2cm] (Patomic1) {\Large Coherent Photoatomic reaction}; ]
				[.\node [sub, xshift=2cm] (Patomic1) {\Large Incoherent Photoatomic reaction}; ]
				[.\node [sub, xshift=2cm] (Patomic1) {\Large Pair Production Photoatomic reaction}; ]
				[.\node [sub, xshift=2cm] (Patomic1) {\Large Photoelectric Photoatomic reaction}; ] 
			]			
			% Photon Scattering Distribution
			[.\node [sub, xshift=-2cm] (Pscat) {\Large Photon Scattering Distribution};
				[.\node [sub, xshift=2.75cm] (Pscat1) {\Large Photon Scattering Distribution}; ]
				[.\node [sub, xshift=2.75cm] (Pscat1) {\Large Photon Scattering Distribution}; ]
				[.\node [sub, xshift=2.75cm] (Pscat1) {\Large Photon Scattering Distribution}; ]
				[.\node [sub, xshift=2.75cm] (Pscat1) {\Large Photon Scattering Distribution}; ]
			]
			% Photon Production Distribution
			[.\node [sub, xshift=-2cm] (Ppro) {\Large Photon Production Distribution}; ]
		] 
        [.\node [dependency, xshift=-4.7cm, yshift=2cm] (core3) {Neutron State}; ]
    ]
\draw [arrow] (core2) -- (Patomic);
\draw [arrow] (core3) -- (Ppro);

% Photon Production Reaction
\node [sub, below of=Ppro,xshift=0cm] (Ppro1) {\Large Photon Production Reaction}; ]
\draw [arrow] (Ppro) -- (Ppro1);
%\draw [arrow] (core1) -- (Ppro1);
%\draw [arrow] (core2) -| (Ppro1);
%\draw [arrow] (core3) -- (Ppro1);

% Photon Data
\tikzset{edge from parent/.style= {level distance=1cm, thick, draw, edge from parent fork left}, grow'=left, <-, parent anchor=west }
\Tree
	[.\node [sub, yshift=-10cm, xshift=2.2cm] (Pdata) {\Large Photon Data};
	    [.\node [dependency, yshift=-9.95cm] (Pdata1) {Photon Data Basic}; ]
		[.\node [dependency, yshift=-9.95cm] (Pdata2) {Photon Data Doppler Broadening}; ] 
		[.\node [dependency, yshift=-9.95cm] (Pdata3) {Photon Data Fluorescence}; ]
    ]
%\node (PData) [sub, yshift=-4cm] {\Large Photon Data};
%\node (PData1) [dependency, left of=PData, xshift=-4cm] {Photon Data Basic};
%\node (PData2) [dependency, above of=PData1] {Photon Data Doppler Broadening};
%\node (PData3) [dependency, below of=PData1] {Photon Data Fluorescence};

%\draw [arrow] (PData1) -- (PData);
%\draw [arrow] (PData2) -- (PData);
%\draw [arrow] (PData3) -- (PData);

% Neutrons, Electrons, and Photons States
%\node (sub2) [electron, below of=start] {Electron State};
%\node (sub1) [neutron, left of=sub2, xshift=-2cm] {Neutron State};
%\node (sub3) [photon, right of=sub2, xshift=2cm] {Photon State};



\end{tikzpicture}
\newpage

% Core Tree
\begin{tikzpicture}[remember picture, overlay, node distance=1.5cm, 
grow'=down, level distance=1.5cm,sibling distance=3.5cm]

\tikzset{edge from parent/.style= 
{thick, draw, edge from parent fork down}
}

% FRENSIE tree
    \node [main, xshift=6.05cm] (0a){\Large FRENSIE}
		% First Branch
        child { node[electron] (1a) {\Large scripts} }
		child { node[electron] (1b) {\Large patches} }
		child {	node[electron] (1c) {\Large \bf packages} 
			% Second Branch
			child { node[photon] (2a) {\large utility} }
			child { node[photon] (2b) {\large transmutation} }
			child { node[photon] (2c) {\large \bf monte carlo}
				%Third Branch 
				child { node[neutron] (3a) {\large source} }
				child { node[neutron] (3b) {\large manager} }
				child { node[neutron] (3c) {\large estimator} }
				child { node[neutron] (3d) {\large doc} }
				child { node[neutron] (3e) {\large \bf core} 
					%Fourth Branch
					child { node[neutron, fill=orange!25] (3a) {\large test} }
					child { node[neutron, fill=orange!25] (3a) {\large \bf src} }
				}
				child { node[neutron] (3f) {\large collision} }
			}
			child { node[photon] (2d) {\large geometry} }
			child { node[photon] (2e) {\large data} }
		}
		child {	node[electron] (1d) {\Large cli} }
		child {	node[electron] (1e) {\Large cmake} }
		child {	node[electron] (1f) {\Large data} }
		child {	node[electron] (1g) {\Large doc} };

	\path (0a) -- (1a);

\end{tikzpicture}

\newpage
% Core Programs
\begin{tikzpicture} [remember picture, overlay, node distance=1.5cm]

% Particle State
\node (start) [main, xshift=6.05cm, yshift=-3cm] {\Large Particle State};
\node (part1) [dependency, above of=start] {Particle State Core};
\node (part2) [dependency, left of=part1, xshift=-2.5cm] {Particle Type};
\node (factory) [main, right of=start, xshift=6cm] {\Large Particle State Factory};

\draw [arrow] (part1) -- (start);
\draw [arrow] (part2) -- (start);
\draw [arrow] (part2) -- (part1);

% Other Programs
\node (shell) [main, above of=part1, xshift=-2cm, yshift=0.8cm] {\Large Simulation Properties};
\node (shell) [main, above of=part1, xshift=-7cm, yshift=0.8cm] {\Large Electron Shell};
\node (unit) [main, above of=part1, xshift=4.2cm, yshift=0.8cm] {\Large Unit Test Harness Extensions};

% Neutrons, Electrons, and Photons States
\node (sub2) [electron, below of=start] {Electron State};
\node (sub1) [neutron, left of=sub2, xshift=-2cm] {Neutron State};
\node (sub3) [photon, right of=sub2, xshift=2cm] {Photon State};

\draw [arrow] (start) -- (sub1);
\draw [arrow] (start) -- (sub2);
\draw [arrow] (start) -- (sub3);
%\draw [arrow] (sub1) -- (factory);
%\draw [arrow] (sub2) -- (factory);
%\draw [arrow] (sub3) -- (factory);
\draw [arrow] (start) -- (factory);

% Photon sub classes
\node (phot1) [dependency, below of=sub3, xshift=2cm, fill=blue!20] {\small Phase Space Dimensions Traits Decl.};
\node (phot2) [dependency, below of=phot1, fill=blue!20] {\small Phase Space Dimensions};
\node (phot3) [dependency, below of=phot2, fill=blue!20] {\small Phase Space Dimensions Traits};

%\draw [arrow] (start) -| (phot1);
\draw [arrow] (sub3) -- (phot1);
\draw [arrow] (phot1) -- (phot2);
\draw [arrow] (phot2) -- (phot3);
\draw [line] (7.85,-3.79) -- (7.85,-9);
\draw [arrow] (7.85,-9) -- (phot3);
\draw [arrow] (7.85,-7.5) -- (phot2);
\draw [arrow] (7.85,-6) -- (phot1);

% Neutron sub classes
\node (neut1) [neutron, below of=sub1, xshift=-2cm, fill=red!20] {Particle Bank};
\node (neut2) [neutron, below of=neut1, yshift=-0.5cm, fill=red!20] {Nuclear Reaction Bank};
\node (neut3) [dependency, left of=neut1, xshift=-3cm, fill=red!10] {\small Nuclear Reaction Type};
\node (neut4) [dependency, above of=neut3, yshift=-0.3cm, fill=red!10] {\small Module Traits};
\node (neut5) [dependency, below of=neut3, yshift=0.3cm, fill=orange!50] {\small Geometry};


\draw [arrow] (start) -| (neut1);
\draw [arrow] (sub1) -- (neut1);
\draw [arrow] (neut1) -- (neut2);
\draw [arrow] (neut3) -- (neut1);
\draw [arrow] (neut4) -- (neut1);
\draw [arrow] (neut5) -- (neut1);
\end{tikzpicture}
\end{document}
