\documentclass{beamer}
\usetheme[white]{Wisconsin}
\usepackage{longtable}
\usepackage{listings}
\usepackage{color}
%% The amssymb package provides various useful mathematical symbols
\usepackage{amssymb}
%% The amsthm package provides extended theorem environments
\usepackage{amsthm} \usepackage{amsmath}
%% \usepackage{eqnarray}
\usepackage[mathcal]{euscript} \usepackage{color}
\usepackage{textcomp}
\usepackage{algorithm,algorithmic}
\usepackage[retainorgcmds]{IEEEtrantools}
\usepackage[absolute,overlay]{textpos}
  \setlength{\TPHorizModule}{1mm}
  \setlength{\TPVertModule}{1mm}
\definecolor{listinggray}{gray}{0.9}
\definecolor{lbcolor}{rgb}{0.9,0.9,0.9}
\lstset{
  backgroundcolor=\color{lbcolor},
  tabsize=4,
  rulecolor=,
  language=c++,
  basicstyle=\scriptsize,
  upquote=true,
  aboveskip={1.5\baselineskip},
  columns=fixed,
  showstringspaces=false,
  extendedchars=true,
  breaklines=true,
  prebreak =
  \raisebox{0ex}[0ex][0ex]{\ensuremath{\hookleftarrow}},
  frame=single,
  showtabs=false,
  showspaces=false,
  showstringspaces=false,
  identifierstyle=\ttfamily,
  keywordstyle=\color[rgb]{0,0,1},
  commentstyle=\color[rgb]{0.133,0.545,0.133},
  stringstyle=\color[rgb]{0.627,0.126,0.941},
}

%% colors
\setbeamercolor{boxheadcolor}{fg=white,bg=UWRed}
\setbeamercolor{boxbodycolor}{fg=black,bg=white}

%%----------------------------------------------------------------------------%%
\author{Luke J. Kersting
    \\ NEEP
    \\ University of Wisconsin - Madison
    \\ FRENSIE Meeting
}

\date{\today}
\title{Update: Electron Mode in FRENSIE}
\begin{document}
\maketitle

%%----------------------------------------------------------------------------%%
\begin{frame}{Outline}

  \begin{block}{Hybrid Multigroup/Continuous-Energy Monte Carlo using Boltzmann-Fokker-Planck}
    \begin{itemize}
      \item Advantages
      \item Boltzmann-Fokker-Planck Equation (BFP)
      \item Modifications to BFB
      \item Solution to Modified BFG
      \item Monte Carlo Method
      \item Adjoint
      \item Other Possible Adjoint Methods
    \end{itemize}
  \end{block}
    

\end{frame}


%%----------------------------------------------------------------------------%%
\begin{frame}{Advantaged}
 
    \begin{itemize}
      \item The same basic multigroup cross-section data can be used for forward and adjoint calculations. 
       
      \item The adjoint transport model is nearly identical to the forward making implementation easy 

      \item The transport equation is generalized for Monte Carlo transport of neutral and charged particles.\\ They implement for electrons and photons. 

    \end{itemize}

\end{frame}


%%----------------------------------------------------------------------------%%
\begin{frame}{Boltzmann-Fokker-Planck Equation}
 
    \begin{eqnarray*}
      \Omega \cdot \nabla \psi + \sigma_t\psi &= & \\
      & & \int_{0}^{\infty}\int_{0}^{2\pi}\int_{-1}^{+1} \sigma_s(E'\rightarrow E,\mu_0)
       \times\psi(\mu',\phi',E')d\mu'd\phi'dE' \\
       & &+ \frac{\alpha}{2}\Big\{ \frac{\partial}{\partial \mu} \Big[(1-\mu^2)\frac{\partial \psi}{\partial \mu}\Big]+ \frac{1}{1-\mu^2}\frac{\partial^2\psi}{\partial\phi^2}\Big\} + \frac{\partial}{\partial E} [S\psi] \\
       & &+ Q
    \end{eqnarray*}

     \begin{itemize}
       \item The Boltzmann Operator treats the large-angled or "smooth" component of the cross-section

      \item The Fokker-Planck Opertor treats the forward-peaked or "singular" component of the cross-section
       
     \end{itemize}

\end{frame}

%%----------------------------------------------------------------------------%%
\begin{frame}{Fokker-Planck Operators}
     
  \begin{block}{Continuous-Scattering Operator}

    $$F_{\alpha}\psi=\frac{\alpha}{2}\Big\{ \frac{\partial}{\partial \mu} 
        \Big[(1-\mu^2)\frac{\partial \psi}{\partial \mu}\Big]+ \frac{1}{1-\mu^2}\frac{\partial^2\psi}{\partial\phi^2}\Big\}$$
      
    \begin{itemize}
      \item Constructed so the mean change in angle cosine per path length is equal to the restricted momentum transfer $$ \Delta\mu/\text{path length} = \text{restricted momentum transfer}$$
    \end{itemize}
  \end{block}

  \begin{block}{Continuous-Slowing Down Operator}

    $$\frac{\partial}{\partial E} [S\psi]$$
      
    \begin{itemize}
      \item Constructed so the mean change in energy per path length is equal to the restricted stopping power $$ \Delta E/\text{path length} = \text{restricted stopping power}$$
    \end{itemize}

  \end{block}

\end{frame}

%%----------------------------------------------------------------------------%%
\begin{frame}{Approximate Angular Fokker-Planck Operator}
 
     Let:  $$\lim_{ \mu_s \to 1}B_{\alpha}\psi = F_{\alpha}\psi $$

     Where:  $$B_{\alpha}\psi=\int_{0}^{2\pi}\int_{-1}^{+1} \sigma_a(E,\mu_0)\psi(\mu',\phi',E)d\mu'd\phi' - \sigma_a\psi$$

    \begin{itemize}
      \item Eigenvalues are equal at limit

      \item High-order eigenvalues become more approximate and are underestimated

      \item Error for higher order flux moments can be ignored if they are large compared to temporal and spatial scale lengths
 
      \item  Holds for condensed history where the scale lengths are large compared to mfp
    \end{itemize}

\end{frame}


%%----------------------------------------------------------------------------%%
\begin{frame}{Legendre Cross-Section Expansion}
 
Expand the cross-sections using Legendre polynomials
$$ \hat{\sigma_s}(E'\to E,\mu_o)=\sum_{l=0}^{L} \frac{2l+1}{4\pi}\sigma_s^{(l)}(E'\to E)P_l(\mu_o)$$

Where:
$$ \sigma_s^{(l)}(E'\to E) = 2\pi\int_{-1}^{+1} \sigma_s(E'\to E),\mu_o)P_l(\mu_o)d\mu_o$$

\end{frame}

%%----------------------------------------------------------------------------%%
\begin{frame}{Hybrid Multigroup/Continuous-Energy Approximation}
 
  \begin{itemize}

    \item Break energy up into $N$ groups such that for group $g$:
    $$ E_{g+1/2} < E < E_{g-1/2} $$ 
    
    \item Radau quadratures are used to get the weighted least-squares fits in energy for: \\~~The Smooth Component Cross-Sections ($\sigma$)
\\~~The Restricted Momentum Transfers ($\alpha$)
\\~~The Restricted Stopping Power ($S$)
    
  \end{itemize}
  
   Replace the parameter, $f$ with $\tilde f$ \\
\[
\tilde{f}(E)=\sum_{g=1}^{N}f_gB_g(E)
~~~\text{Where}~~~  B_g(E) =
\begin{cases} 
1 & E \in (E_{g+1/2},E_{g-1/2}) \\ 
0 & Otherwise
\end{cases}
\] 

   $f_g$ is the weighted group average of $f(E)$ using Radau quadratures

\end{frame}

%%----------------------------------------------------------------------------%%
\begin{frame}{Multigroup/Continuous Energy BFP Equation}
 
    \begin{eqnarray*}
      \Omega \cdot \nabla \psi + \tilde{\sigma}_t\psi &= & \\
      & & \int_{E}^{E_{1/2}}\int_{0}^{2\pi}\int_{-1}^{+1} \tilde{\sigma}_s(E'\rightarrow E,\mu_0)
       \psi(\mu',\phi',E')d\mu'd\phi'dE' \\
       & &+ \int_{0}^{2\pi}\int_{-1}^{+1} \tilde{\sigma}_{\alpha}(\mu_o) \psi(\mu',\phi')d\mu'd\phi'  - \tilde{\sigma}_{\alpha}(\mu_o)\psi\\
       & &+ \frac{\partial}{\partial E} [\tilde{S}\psi] + Q
    \end{eqnarray*}

     \begin{itemize}
       \item The Boltzmann Operator reduces to Standard Multigroup method.

      \item Exponential distribution of path lengths (compared fixed path length for condensed history).
   
      \item Accuracy depends on:  \# of groups, Order of Legendre expansion, $\mu_s$
       
     \end{itemize}

\end{frame}

%%----------------------------------------------------------------------------%%
\begin{frame}{Collision Algorithm: Location and Energy}

Let $E_p$ be the energy of a particle in group $g$

  \begin{itemize}
    \item The total group cross-section is the sum of the smooth-component Boltzmann and continuous-scattering cross-sections: $$~~~ \sigma_g^{total} = \sigma_{t,g} + \sigma_{\alpha,g} ~~~\text{Where}~~~ \sigma_{\alpha,g}=\frac{\alpha}{1-\mu_s}$$


       \item $\sigma_g^{total}$ is used to find the distance to next collision, $D_c$

      \item $D_c$ is compared to the distance to material, $D_m$, and distance to energy, $D_e = \frac{E_p - E_{g+1/2}}{S_g}$
   
      \item The new energy is: $$E_p^{new} = E_p^{old} - S_gD_c$$
       
     \end{itemize}

\end{frame}

%%----------------------------------------------------------------------------%%
\begin{frame}{Collision Algorithm: Reaction}

Can either have a smooth-component Boltzmann or continuous-scattering reaction with probabilities: $$ P_B=\frac{\sigma_{t,g}}{\sigma_g^{total} }~~~\text{and}~~~
                     P_{\alpha}=\frac{\sigma_{\alpha,g}}{\sigma_g^{total} }$$

  \begin{itemize}
    \item If $P_{\alpha}$ is selected a new direction for the particle is randomly sampled based on a polar scattering angle with cosine equal to $\mu_s$.
    
    \item  If $P_B$ is selected the particle is removed and $M$ new particle are generated at the collision site.

    \item Multiplication Factor
   $$ M = \frac{1}{\sigma_g^{total}} \int_{E_{g+1/2}}^{E_{g-1/2}}  \sigma_s^{(0)}(E'\to E)dE'=\frac{1}{\sigma_{t,g}}\sum_{k=g}^{N}\sigma_{s,g\to k}^{(0)}$$

       
     \end{itemize}

\end{frame}


%%----------------------------------------------------------------------------%%
\begin{frame}{Collision Algorithm: Reaction (Continued)}

Average M must be preserved

  \begin{itemize}
    \item Let $M =\text{Integer} + \text{Remainder} = I + R$
    
    \item  Create $I$ or $I+1$ particles with probability $1.0-R$ or $R$.
    
  \end{itemize}

Energy

  \begin{itemize}
    \item Particles generated in group $g$ has an energy range of $E_{g+1/2} < E< E_{g-1/2}$
    
    \item  Randomly sample energy from a uniform distribution.
    
  \end{itemize}

Angle

  \begin{itemize}
    \item Sample angle based on the discrete Radau distributions.
    
    \item  Separate Radau distribution for each smooth-component Boltzmann group-to-group transfer.
    
  \end{itemize}

\end{frame}


%%----------------------------------------------------------------------------%%
\begin{frame}{Adjoint Multigroup/Continuous Energy BFP Equation}
 
    \begin{eqnarray*}
      -\Omega \cdot \nabla \psi^{\dagger} + \tilde{\sigma}_t\psi^{\dagger} = & \\
      & \int_{E_{N+1/2}}^{E}\int_{0}^{2\pi}\int_{-1}^{+1} \tilde{\sigma}_s(E\rightarrow E',\mu_0)
       \psi^{\dagger}({\bf r},\mu',\phi',E')d\mu'd\phi'dE' \\
       &+ \int_{0}^{2\pi}\int_{-1}^{+1} \tilde{\sigma}_{\alpha}(\mu_o) \psi^{\dagger}(\mu',\phi')d\mu'd\phi'  - \tilde{\sigma}_{\alpha}(\mu_o)\psi^{\dagger}\\
       &- \frac{\partial}{\partial E} [\tilde{S}\psi^{\dagger}]  + \frac{\partial S}{\partial E} \psi^{\dagger} + Q^{\dagger}
    \end{eqnarray*}

     \begin{itemize}
       \item Let the dot product be: $$ [f,h]=\sum_{g=1}^{N}f_gh_g\frac{1}{\Delta E_g} $$

      \item The adjoint cross-section is then: $$ \sigma_{s,k\to g}^{\dagger(l)} = \sigma_{s,g\to k}^{(l)} \frac{\Delta E_g}{\Delta E_k} $$
       
     \end{itemize}

\end{frame}


%%----------------------------------------------------------------------------%%
\begin{frame}{Other Possible Adjoint Methods}

     \begin{itemize}
       \item 1980 - Adjoint Electron Transport in the CSDA
           \begin{itemize}
             \item Goudsmit and Saunderson Scattering
           \end{itemize}
                 
       \item 1995 - Adjoint Elecrton-Photon Tansport using BFS in ITS
           \begin{itemize}
             \item Multigroup/Continuous Energy
           \end{itemize}
           
       \item 1996 - Adjoint Multigroup/Continuous Energy BFP Equation

      \item 2005 - Generalized Particle for Couple Adjoint $\gamma$ - $e^-$ - $e^+$ Transport
           \begin{itemize}
             \item CSDA using Moli\`{e}re's multiple scattering
           \end{itemize}
       
     \end{itemize}



\end{frame}


\end{document}
