\documentclass{beamer}
\usetheme[white]{Wisconsin}
\usepackage{longtable}
\usepackage{listings}
\usepackage{color}
%% The amssymb package provides various useful mathematical symbols
\usepackage{amssymb}
%% The amsthm package provides extended theorem environments
\usepackage{amsthm} \usepackage{amsmath}
%% \usepackage{eqnarray}
\usepackage[mathcal]{euscript} \usepackage{color}
\usepackage{textcomp}
\usepackage{algorithm,algorithmic}
\usepackage[retainorgcmds]{IEEEtrantools}
\usepackage[absolute,overlay]{textpos}
  \setlength{\TPHorizModule}{1mm}
  \setlength{\TPVertModule}{1mm}
\definecolor{listinggray}{gray}{0.9}
\definecolor{lbcolor}{rgb}{0.9,0.9,0.9}
\lstset{
  backgroundcolor=\color{lbcolor},
  tabsize=4,
  rulecolor=,
  language=c++,
  basicstyle=\scriptsize,
  upquote=true,
  aboveskip={1.5\baselineskip},
  columns=fixed,
  showstringspaces=false,
  extendedchars=true,
  breaklines=true,
  prebreak =
  \raisebox{0ex}[0ex][0ex]{\ensuremath{\hookleftarrow}},
  frame=single,
  showtabs=false,
  showspaces=false,
  showstringspaces=false,
  identifierstyle=\ttfamily,
  keywordstyle=\color[rgb]{0,0,1},
  commentstyle=\color[rgb]{0.133,0.545,0.133},
  stringstyle=\color[rgb]{0.627,0.126,0.941},
}

%% colors
\setbeamercolor{boxheadcolor}{fg=white,bg=UWRed}
\setbeamercolor{boxbodycolor}{fg=black,bg=white}

%%----------------------------------------------------------------------------%%
\author{Luke J. Kersting
    \\ NEEP
    \\ University of Wisconsin - Madison
    \\ SNL Meeting
}

\date{\today}
\title{Electron Mode in FRENSIE}
\begin{document}
\maketitle

%%----------------------------------------------------------------------------%%
\begin{frame}{Electron Transport in FRENSIE}

  \begin{block}{Forward Mode}
    \begin{itemize}
      \item Condensed History
      \item Secondary Particles
      \item Atomic Relaxation
      \item Simulation of hard electron transport events
      \begin{itemize}
         \item Atomic excitation
         \item Hard elastic scattering
         \item Electroionization
         \item Bremmstrahlung
      \end{itemize}
    \end{itemize}
  \end{block}
    
  \begin{block}{Adjoint Mode}
    \begin{itemize}
      \item Hybrid Multigroup/Continuous-Energy Monte Carlo using Boltzmann-Fokker-Planck Equation (BFP)
    \end{itemize}    
  \end{block}  

\end{frame}


%%----------------------------------------------------------------------------%%
\begin{frame}{Electron Transport in Monte Carlo Codes}

  \begin{block}{MCNP}
    \begin{itemize}
      \item Historically has only used a condensed-history approched with Goudsmit-Saunderson mutiple scattering techniques.
      \item MCNP6 implemented a single-event method for energies below 1 keV, 
              were the condensed-history method no longer holds.
    \end{itemize}
  \end{block}
    
  \begin{block}{Penelope}
    \begin{itemize}
      \item Implenments a mixed method that simulates soft (condensed-history) 
              events below a cutoff energy/angle and hard (single-events) above.
      \item Uses Goudsmit-Saunderson Multiple Scattering
    \end{itemize}    
  \end{block}

  \begin{block}{EGS}
    \begin{itemize}
      \item Condensed History Method
      \item Historically used Moli\`ere Multiple Scattering Theory
      \item EGS5 implemented Goudsmit-Saunderson Multiple Scattering to take into account spin and relativistic effects needed in the MeV range
    \end{itemize}    
  \end{block}
    

\end{frame}


%%----------------------------------------------------------------------------%%
\begin{frame}{Electron Mode}

  \begin{block}{Frensie}
    \begin{itemize}
      \item Hard events implemented using cross-sectional data from MCNP6
      \item Condensed history method will be chosen in conjunction with an adjoint method
      \item Ultimately hope to implement a mixed method for forward transport
    \end{itemize}    
  \end{block}
  
  \begin{block}{Current Capabilities}
    \begin{itemize}
      \item Single Scattering Events from 100 GeV to 10 eV
      \item Elastic, Bremsstrahlung, Electroionization, Atomic Excitation 
      \item Secondary particles created, but photons not tracked
      \item Atomic relaxation implemented
    \end{itemize}
  \end{block}
    
  \begin{block}{Problems}
    \begin{itemize}
      \item Absorption at low energies
      \item Negative energy from Electroionization
    \end{itemize}    
  \end{block}  

\end{frame}

%%----------------------------------------------------------------------------%%
\begin{frame}{Atomic Excitation}

  \begin{block}{Reaction}
    \begin{itemize}
      \item There is no angular deflection.
      \item There are no secondary particles.
    \end{itemize}
  \end{block}  

~~\\
  \begin{block}{Implementation}
    \begin{itemize}
      \item Energy dependent electron energy loss are tabulated in ACE tables.
      \item No sampling is required for this process.   
    \end{itemize}
  \end{block}  

\end{frame}

%%----------------------------------------------------------------------------%%
\begin{frame}{Hard Elastic Scattering}
  
  \begin{block}{Reaction}
    \begin{itemize}
      \item There is no energy loss.
      \item There are no secondary particles.
    \end{itemize}
  \end{block}  
      
  \begin{block}{Implementation}
    \begin{itemize}
      \item ACE tables provide histogram CDF of the outgoing angle cosine, \textmu, 
            for $14-16$ energy groups.
      \item for $\mu > 0.999999$ an analytical function, $f(\mu)$, derived from Moli\`ere's screening factor is used to compute the scattering angle
    \end{itemize}

  \begin{equation*}
    f(\mu) = \frac{A}{(\eta + 1 - \mu)^2}
  \end{equation*}

  \begin{equation*}
    \eta(E,Z) = \frac{1}{4}\left(\frac{\alpha mc}{0.885p}\right)^2 Z^{2/3}[1.13+3.76(\alpha Z/\beta)^2]
  \end{equation*}
  
    \end{block}  


\end{frame}

%%----------------------------------------------------------------------------%%
\begin{frame}{Electroionization}

  \begin{block}{Reaction}
    \begin{itemize}
      \item The subshell is directly sampled.
      \item A knock-on electron is ejected.
      \item The incident electron energy is reduced by the $E_{knock} + E_{binding}$.
      \end{itemize}
  \end{block}  
      
  \begin{block}{Implementation}
    \begin{itemize}
      \item ACE tables provide CDF of the knock-on energy, $E_{knock}$, based on the incident electron energy.
      \item Conservation of momentum is used to find the scattering and ejection angles (which are sampled independently).
      \item The shell vacancy is handled using atomic relaxation data. 
    \end{itemize}
  \end{block}  

\end{frame}

%%----------------------------------------------------------------------------%%
\begin{frame}{Electroionization Scattering Angle}

{\large Conservation of Momentum}
  \begin{align}
    (p_{knock}c + p_{a}c)^2 = & (pc)^2 + (p'c)^2 - 2pp'cos(\theta) \nonumber \\
   cos(\theta) = & \frac{(pc)^2 + (p'c)^2 - (p_{knock}c)^2}{2pp'}  \nonumber
  \end{align}

{\large Conservation of Energy}
  \begin{equation*}
    (T + m_ec^2) + (M_ac^2) =(T' + m_ec^2) + (T_a + M_ac^2 + T_{knock} + m_ec^2) + E_{Binding}
  \end{equation*}
Assume the binding energy is negligible 
  \begin{equation*}
    T = T' + T_{knock}
  \end{equation*}
  Solving you obtain:
  $$ cos(\theta) = \frac{T'}{T}\frac{p}{p'}~~~~\text{and}~~~~
  cos(\phi)=\frac{T_{knock}}{T}\frac{p}{p_{knock}}$$


\end{frame}

%%----------------------------------------------------------------------------%%
\begin{frame}{Sampling Electroionization}
  
{\large The original sampling routine implemented in FRENSIE differed slightly from MCNP6 which caused the sampling of negative electrons energies. } \\ 
~~\\

  \begin{itemize}
    \item ACE tables provide CDF of the knock-on energy, $E_{knock}$, based on the incident electron energy.
    \item The original implementation randomly selected whether to sample the upper or lower energy bin.
    \item A correlated sample must be made to avoid non physical values.
  \end{itemize}

\end{frame}


%%----------------------------------------------------------------------------%%
\begin{frame}{Bremmstrahlung}
  \begin{block}{Reaction}
  \begin{itemize}
    \item A photon is ejected.
    \item ACE tables provide CDF of the photon energy, $E_{\gamma}$, based on the incident electron energy.
    \item The incident electron energy is reduced by the $E_{\gamma}$.
    \item The electron direction is assumed to be essentially unchanged.
      \end{itemize}
  \end{block}  
      
  \begin{block}{Implementation}
    \begin{itemize}
    \item An analytical dipole function, $p(\mu)$, is used to sample the direction of the outgoing photon.
    \item MCNP6 also uses a table based scheme from their condensed history method.
  \end{itemize}
  \begin{equation*}
    p(\mu)d\mu = \frac{(1-\beta^2)}{2(1-\beta\mu)^2}d\mu
  \end{equation*}
  \end{block}  

\end{frame}

%%----------------------------------------------------------------------------%%
\begin{frame}{Known Problems}
  \begin{block}{Absorption at low energies}
  \begin{itemize}
    \item At energies near the cutoff (10 eV) the reaction cross section is dominated by elastic scattering (by order $10^7$ for H)
    \item It is unlikely the electron will scatter below the cutoff energy 
    \item A temporary fix is to raise the cutoff energy (to 15eV for H) to prevent indefinite elastic scattering
    \item MCNP notes this problem and suggests a minimum cutoff energy of 20eV
  \end{itemize}
\end{block}
\end{frame}

  %%----------------------------------------------------------------------------%%
\begin{frame}{Adjoint Mode Outline}

  \begin{block}{Hybrid Multigroup/Continuous-Energy BFP}
    \begin{itemize}
      \item Advantages
      \item Boltzmann-Fokker-Planck Equation (BFP)
      \item Modifications to BFB
      \item Solution to Modified BFG
      \item Monte Carlo Method
      \item Adjoint
      \item Other Possible Adjoint Methods
    \end{itemize}
  \end{block}
    

\end{frame}

%%----------------------------------------------------------------------------%%
\begin{frame}{Advantaged}
 
    \begin{itemize}
      \item The same basic multigroup cross-section data can be used for forward and adjoint calculations. 
       
      \item The adjoint transport model is nearly identical to the forward making implementation easy 

      \item The transport equation is generalized for Monte Carlo transport of neutral and charged particles.\\ They implement for electrons and photons. 

    \end{itemize}

\end{frame}


%%----------------------------------------------------------------------------%%
\begin{frame}{Boltzmann-Fokker-Planck Equation}
 
    \begin{eqnarray*}
      \Omega \cdot \nabla \psi + \sigma_t\psi &= & \\
      & & \int_{0}^{\infty}\int_{0}^{2\pi}\int_{-1}^{+1} \sigma_s(E'\rightarrow E,\mu_0)
       \times\psi(\mu',\phi',E')d\mu'd\phi'dE' \\
       & &+ \frac{\alpha}{2}\Big\{ \frac{\partial}{\partial \mu} \Big[(1-\mu^2)\frac{\partial \psi}{\partial \mu}\Big]+ \frac{1}{1-\mu^2}\frac{\partial^2\psi}{\partial\phi^2}\Big\} + \frac{\partial}{\partial E} [S\psi] \\
       & &+ Q
    \end{eqnarray*}

     \begin{itemize}
       \item The Boltzmann Operator treats the large-angled or "smooth" component of the cross-section

      \item The Fokker-Planck Opertor treats the forward-peaked or "singular" component of the cross-section
       
     \end{itemize}

\end{frame}

%%----------------------------------------------------------------------------%%
\begin{frame}{Fokker-Planck Operators}
     
  \begin{block}{Continuous-Scattering Operator}

    $$F_{\alpha}\psi=\frac{\alpha}{2}\Big\{ \frac{\partial}{\partial \mu} 
        \Big[(1-\mu^2)\frac{\partial \psi}{\partial \mu}\Big]+ \frac{1}{1-\mu^2}\frac{\partial^2\psi}{\partial\phi^2}\Big\}$$
      
    \begin{itemize}
      \item Constructed so the mean change in angle cosine per path length is equal to the restricted momentum transfer $$ \Delta\mu/\text{path length} = \text{restricted momentum transfer}$$
    \end{itemize}
  \end{block}

  \begin{block}{Continuous-Slowing Down Operator}

    $$\frac{\partial}{\partial E} [S\psi]$$
      
    \begin{itemize}
      \item Constructed so the mean change in energy per path length is equal to the restricted stopping power $$ \Delta E/\text{path length} = \text{restricted stopping power}$$
    \end{itemize}

  \end{block}

\end{frame}

%%----------------------------------------------------------------------------%%
\begin{frame}{Approximate Angular Fokker-Planck Operator}
 
     Let:  $$\lim_{ \mu_s \to 1}B_{\alpha}\psi = F_{\alpha}\psi $$

     Where:  $$B_{\alpha}\psi=\int_{0}^{2\pi}\int_{-1}^{+1} \sigma_a(E,\mu_0)\psi(\mu',\phi',E)d\mu'd\phi' - \sigma_a\psi$$

    \begin{itemize}
      \item Eigenvalues are equal at limit

      \item High-order eigenvalues become more approximate and are underestimated

      \item Error for higher order flux moments can be ignored if they are large compared to temporal and spatial scale lengths
 
      \item  Holds for condensed history where the scale lengths are large compared to mfp
    \end{itemize}

\end{frame}


%%----------------------------------------------------------------------------%%
\begin{frame}{Legendre Cross-Section Expansion}
 
Expand the cross-sections using Legendre polynomials
$$ \hat{\sigma_s}(E'\to E,\mu_o)=\sum_{l=0}^{L} \frac{2l+1}{4\pi}\sigma_s^{(l)}(E'\to E)P_l(\mu_o)$$

Where:
$$ \sigma_s^{(l)}(E'\to E) = 2\pi\int_{-1}^{+1} \sigma_s(E'\to E),\mu_o)P_l(\mu_o)d\mu_o$$

\end{frame}

%%----------------------------------------------------------------------------%%
\begin{frame}{Hybrid Multigroup/Continuous-Energy Approximation}
 
  \begin{itemize}

    \item Break energy up into $N$ groups such that for group $g$:
    $$ E_{g+1/2} < E < E_{g-1/2} $$ 
    
    \item Radau quadratures are used to get the weighted least-squares fits in energy for: \\~~The Smooth Component Cross-Sections ($\sigma$)
\\~~The Restricted Momentum Transfers ($\alpha$)
\\~~The Restricted Stopping Power ($S$)
    
  \end{itemize}
  
   Replace the parameter, $f$ with $\tilde f$ \\
\[
\tilde{f}(E)=\sum_{g=1}^{N}f_gB_g(E)
~~~\text{Where}~~~  B_g(E) =
\begin{cases} 
1 & E \in (E_{g+1/2},E_{g-1/2}) \\ 
0 & Otherwise
\end{cases}
\] 

   $f_g$ is the weighted group average of $f(E)$ using Radau quadratures

\end{frame}

%%----------------------------------------------------------------------------%%
\begin{frame}{Multigroup/Continuous Energy BFP Equation}
 
    \begin{eqnarray*}
      \Omega \cdot \nabla \psi + \tilde{\sigma}_t\psi &= & \\
      & & \int_{E}^{E_{1/2}}\int_{0}^{2\pi}\int_{-1}^{+1} \tilde{\sigma}_s(E'\rightarrow E,\mu_0)
       \psi(\mu',\phi',E')d\mu'd\phi'dE' \\
       & &+ \int_{0}^{2\pi}\int_{-1}^{+1} \tilde{\sigma}_{\alpha}(\mu_o) \psi(\mu',\phi')d\mu'd\phi'  - \tilde{\sigma}_{\alpha}(\mu_o)\psi\\
       & &+ \frac{\partial}{\partial E} [\tilde{S}\psi] + Q
    \end{eqnarray*}

     \begin{itemize}
       \item The Boltzmann Operator reduces to Standard Multigroup method.

      \item Exponential distribution of path lengths (compared fixed path length for condensed history).
   
      \item Accuracy depends on:  \# of groups, Order of Legendre expansion, $\mu_s$
       
     \end{itemize}

\end{frame}

%%----------------------------------------------------------------------------%%
\begin{frame}{Collision Algorithm: Location and Energy}

Let $E_p$ be the energy of a particle in group $g$

  \begin{itemize}
    \item The total group cross-section is the sum of the smooth-component Boltzmann and continuous-scattering cross-sections: $$~~~ \sigma_g^{total} = \sigma_{t,g} + \sigma_{\alpha,g} ~~~\text{Where}~~~ \sigma_{\alpha,g}=\frac{\alpha}{1-\mu_s}$$


       \item $\sigma_g^{total}$ is used to find the distance to next collision, $D_c$

      \item $D_c$ is compared to the distance to material, $D_m$, and distance to energy, $D_e = \frac{E_p - E_{g+1/2}}{S_g}$
   
      \item The new energy is: $$E_p^{new} = E_p^{old} - S_gD_c$$
       
     \end{itemize}

\end{frame}

%%----------------------------------------------------------------------------%%
\begin{frame}{Collision Algorithm: Reaction}

Can either have a smooth-component Boltzmann or continuous-scattering reaction with probabilities: $$ P_B=\frac{\sigma_{t,g}}{\sigma_g^{total} }~~~\text{and}~~~
                     P_{\alpha}=\frac{\sigma_{\alpha,g}}{\sigma_g^{total} }$$

  \begin{itemize}
    \item If $P_{\alpha}$ is selected a new direction for the particle is randomly sampled based on a polar scattering angle with cosine equal to $\mu_s$.
    
    \item  If $P_B$ is selected the particle is removed and $M$ new particle are generated at the collision site.

    \item Multiplication Factor
   $$ M = \frac{1}{\sigma_g^{total}} \int_{E_{g+1/2}}^{E_{g-1/2}}  \sigma_s^{(0)}(E'\to E)dE'=\frac{1}{\sigma_{t,g}}\sum_{k=g}^{N}\sigma_{s,g\to k}^{(0)}$$

       
     \end{itemize}

\end{frame}


%%----------------------------------------------------------------------------%%
\begin{frame}{Collision Algorithm: Reaction (Continued)}

Average M must be preserved

  \begin{itemize}
    \item Let $M =\text{Integer} + \text{Remainder} = I + R$
    
    \item  Create $I$ or $I+1$ particles with probability $1.0-R$ or $R$.
    
  \end{itemize}

Energy

  \begin{itemize}
    \item Particles generated in group $g$ has an energy range of $E_{g+1/2} < E< E_{g-1/2}$
    
    \item  Randomly sample energy from a uniform distribution.
    
  \end{itemize}

Angle

  \begin{itemize}
    \item Sample angle based on the discrete Radau distributions.
    
    \item  Separate Radau distribution for each smooth-component Boltzmann group-to-group transfer.
    
  \end{itemize}

\end{frame}


%%----------------------------------------------------------------------------%%
\begin{frame}{Adjoint Multigroup/Continuous Energy BFP Equation}
 
    \begin{eqnarray*}
      -\Omega \cdot \nabla \psi^{\dagger} + \tilde{\sigma}_t\psi^{\dagger} = & \\
      & \int_{E_{N+1/2}}^{E}\int_{0}^{2\pi}\int_{-1}^{+1} \tilde{\sigma}_s(E\rightarrow E',\mu_0)
       \psi^{\dagger}({\bf r},\mu',\phi',E')d\mu'd\phi'dE' \\
       &+ \int_{0}^{2\pi}\int_{-1}^{+1} \tilde{\sigma}_{\alpha}(\mu_o) \psi^{\dagger}(\mu',\phi')d\mu'd\phi'  - \tilde{\sigma}_{\alpha}(\mu_o)\psi^{\dagger}\\
       &- \frac{\partial}{\partial E} [\tilde{S}\psi^{\dagger}]  + \frac{\partial S}{\partial E} \psi^{\dagger} + Q^{\dagger}
    \end{eqnarray*}

     \begin{itemize}
       \item Let the dot product be: $$ [f,h]=\sum_{g=1}^{N}f_gh_g\frac{1}{\Delta E_g} $$

      \item The adjoint cross-section is then: $$ \sigma_{s,k\to g}^{\dagger(l)} = \sigma_{s,g\to k}^{(l)} \frac{\Delta E_g}{\Delta E_k} $$
       
     \end{itemize}

\end{frame}


%%----------------------------------------------------------------------------%%
\begin{frame}{Other Possible Adjoint Methods}

     \begin{itemize}
       \item 1980 - Adjoint Electron Transport in the CSDA
           \begin{itemize}
             \item Goudsmit and Saunderson Scattering
           \end{itemize}
                 
       \item 1995 - Adjoint Elecrton-Photon Tansport using BFS in ITS
           \begin{itemize}
             \item Multigroup/Continuous Energy
           \end{itemize}
           
       \item 1996 - Adjoint Multigroup/Continuous Energy BFP Equation

      \item 2005 - Generalized Particle for Couple Adjoint $\gamma$ - $e^-$ - $e^+$ Transport
           \begin{itemize}
             \item CSDA using Moli\`{e}re's multiple scattering
           \end{itemize}
       
     \end{itemize}



\end{frame}

\end{document}
