\documentclass{beamer}
\usetheme[white]{Wisconsin}
\usepackage{longtable}
\usepackage{graphicx}
\usepackage{listings}
\usepackage{color}
%% The amssymb package provides various useful mathematical symbols
\usepackage{amssymb}
%% The amsthm package provides extended theorem environments
\usepackage{amsthm} \usepackage{amsmath}
%% \usepackage{eqnarray}
\usepackage[mathcal]{euscript} \usepackage{color}
\usepackage{textcomp}
\usepackage{algorithm,algorithmic}
\usepackage[retainorgcmds]{IEEEtrantools}
\usepackage[absolute,overlay]{textpos}
  \setlength{\TPHorizModule}{1mm}
  \setlength{\TPVertModule}{1mm}
\definecolor{listinggray}{gray}{0.9}
\definecolor{lbcolor}{rgb}{0.9,0.9,0.9}
\lstset{
  backgroundcolor=\color{lbcolor},
  tabsize=4,
  rulecolor=,
  language=c++,
  basicstyle=\scriptsize,
  upquote=true,
  aboveskip={1.5\baselineskip},
  columns=fixed,
  showstringspaces=false,
  extendedchars=true,
  breaklines=true,
  prebreak =
  \raisebox{0ex}[0ex][0ex]{\ensuremath{\hookleftarrow}},
  frame=single,
  showtabs=false,
  showspaces=false,
  showstringspaces=false,
  identifierstyle=\ttfamily,
  keywordstyle=\color[rgb]{0,0,1},
  commentstyle=\color[rgb]{0.133,0.545,0.133},
  stringstyle=\color[rgb]{0.627,0.126,0.941},
}

%% colors
\setbeamercolor{boxheadcolor}{fg=white,bg=UWRed}
\setbeamercolor{boxbodycolor}{fg=black,bg=white}

%%----------------------------------------------------------------------------%%
\author{Luke J. Kersting
    \\ NEEP
    \\ University of Wisconsin - Madison
    \\ SNL Meeting
}

\date{\today}
\title{Electron Mode in FRENSIE}
\begin{document}
\maketitle

%%----------------------------------------------------------------------------%%
\begin{frame}{Electron Transport in FRENSIE}

  \begin{block}{Forward Mode}
    \begin{itemize}
      \item Condensed History
      \item Secondary Particles
      \item Atomic Relaxation
      \item Simulation of hard electron transport events
      \begin{itemize}
         \item Atomic excitation
         \item Hard elastic scattering
         \item Electroionization
         \item Bremmstrahlung
      \end{itemize}
    \end{itemize}
  \end{block}
    
  \begin{block}{Adjoint Mode}
    \begin{itemize}
      \item Hybrid Multigroup/Continuous-Energy Monte Carlo using Boltzmann-Fokker-Planck Equation (BFP)
      \item Other Possible Adjoint Methods
    \end{itemize}    
  \end{block}  

\end{frame}


%%----------------------------------------------------------------------------%%
\begin{frame}{Electron Transport in Monte Carlo Codes}

  \begin{block}{MCNP}
    \begin{itemize}
      \item Historically has only used a condensed-history approched with Goudsmit-Saunderson mutiple scattering techniques.
      \item MCNP6 implemented a single-event method for energies below 1 keV, 
              were the condensed-history method no longer holds.
    \end{itemize}
  \end{block}
    
  \begin{block}{Penelope}
    \begin{itemize}
      \item Implenments a mixed method that simulates soft (condensed-history) 
              events below a cutoff energy/angle and hard (single-events) above.
      \item Uses Goudsmit-Saunderson Multiple Scattering
    \end{itemize}    
  \end{block}

  \begin{block}{EGS}
    \begin{itemize}
      \item Condensed History Method
      \item Historically used Moli\`ere Multiple Scattering Theory
      \item EGS5 implemented Goudsmit-Saunderson Multiple Scattering to take into account spin and relativistic effects needed in the MeV range
    \end{itemize}    
  \end{block}
    

\end{frame}


%%----------------------------------------------------------------------------%%
\begin{frame}{Electron Mode}

  \begin{block}{Frensie}
    \begin{itemize}
      \item Hard events implemented using cross-sectional data from MCNP6
      \item Condensed history method will be chosen in conjunction with an adjoint method
      \item Ultimately hope to implement a mixed method for forward transport
    \end{itemize}    
  \end{block}
  
  \begin{block}{Current Capabilities}
    \begin{itemize}
      \item Single Scattering Events from 100 GeV to 10 eV
      \item Elastic, Bremsstrahlung, Electroionization, Atomic Excitation 
      \item Secondary particles created, but photons not tracked
      \item Atomic relaxation implemented
    \end{itemize}
  \end{block}
    
  \begin{block}{Known Issues}
    \begin{itemize}
      \item Absorption at low energies
      \item Negative energy from Electroionization
    \end{itemize}    
  \end{block}  

\end{frame}

%%----------------------------------------------------------------------------%%
\begin{frame}{Atomic Excitation}

  \begin{block}{Reaction}
    \begin{itemize}
      \item There is no angular deflection.
      \item There are no secondary particles.
    \end{itemize}
  \end{block}  

~~\\
  \begin{block}{Implementation}
    \begin{itemize}
      \item Energy dependent electron energy loss are tabulated in ACE tables.
      \item No sampling is required for this process.   
    \end{itemize}
  \end{block}  

\end{frame}

%%----------------------------------------------------------------------------%%
\begin{frame}{Hard Elastic Scattering}
  
  \begin{block}{Reaction}
    \begin{itemize}
      \item There is no energy loss.
      \item There are no secondary particles.
    \end{itemize}
  \end{block}  
      
  \begin{block}{Implementation}
    \begin{itemize}
      \item ACE tables provide histogram CDF of the outgoing angle cosine, \textmu, 
            for $14-16$ energy groups.
      \item for $\mu > 0.999999$ an analytical function, $f(\mu)$, derived from Moli\`ere's screening factor is used to compute the scattering angle
    \end{itemize}

  \begin{equation*}
    f(\mu) = \frac{A}{(\eta + 1 - \mu)^2}
  \end{equation*}

  \begin{equation*}
    \eta(E,Z) = \frac{1}{4}\left(\frac{\alpha mc}{0.885p}\right)^2 Z^{2/3}[1.13+3.76(\alpha Z/\beta)^2]
  \end{equation*}
  
    \end{block}  


\end{frame}

%%----------------------------------------------------------------------------%%
\begin{frame}{Electroionization}

  \begin{block}{Reaction}
    \begin{itemize}
      \item The subshell is directly sampled.
      \item A knock-on electron is ejected.
      \item The incident electron energy is reduced by the $E_{knock} + E_{binding}$.
      \end{itemize}
  \end{block}  
      
  \begin{block}{Implementation}
    \begin{itemize}
      \item ACE tables provide CDF of the knock-on energy, $E_{knock}$, based on the incident electron energy.
      \item Conservation of momentum is used to find the scattering and ejection angles (which are sampled independently).
      \item The shell vacancy is handled using atomic relaxation data. 
    \end{itemize}
  \end{block}  

\end{frame}

%%----------------------------------------------------------------------------%%
\begin{frame}{Electroionization Scattering Angle}

{\large Conservation of Momentum}
  \begin{align}
    (p_{knock}c + p_{a}c)^2 = & (pc)^2 + (p'c)^2 - 2pp'cos(\theta) \nonumber \\
   cos(\theta) = & \frac{(pc)^2 + (p'c)^2 - (p_{knock}c)^2}{2pp'}  \nonumber
  \end{align}

{\large Conservation of Energy}
  \begin{equation*}
    (T + m_ec^2) + (M_ac^2) =(T' + m_ec^2) + (T_a + M_ac^2 + T_{knock} + m_ec^2) + E_{Binding}
  \end{equation*}
Assume the binding energy is negligible 
  \begin{equation*}
    T = T' + T_{knock}
  \end{equation*}
  Solving you obtain:
  $$ cos(\theta) = \frac{T'}{T}\frac{p}{p'}~~~~\text{and}~~~~
  cos(\phi)=\frac{T_{knock}}{T}\frac{p}{p_{knock}}$$


\end{frame}

%%----------------------------------------------------------------------------%%
\begin{frame}{Sampling Electroionization}
  
{\large The original sampling routine implemented in FRENSIE differed slightly from MCNP6 which caused the sampling of negative electrons energies. } \\ 
~~\\

  \begin{itemize}
    \item ACE tables provide CDF of the knock-on energy, $E_{knock}$, based on the incident electron energy.
    \item The original implementation randomly selected whether to sample the upper or lower energy bin.
    \item A correlated sample must be made to avoid non physical values.
  \end{itemize}

\end{frame}


%%----------------------------------------------------------------------------%%
\begin{frame}{Bremmstrahlung}
  \begin{block}{Reaction}
  \begin{itemize}
    \item A photon is ejected.
    \item ACE tables provide CDF of the photon energy, $E_{\gamma}$, based on the incident electron energy.
    \item The incident electron energy is reduced by the $E_{\gamma}$.
    \item The electron direction is assumed to be essentially unchanged.
      \end{itemize}
  \end{block}  
      
  \begin{block}{Implementation}
    \begin{itemize}
    \item An analytical dipole function, $p(\mu)$, is used to sample the direction of the outgoing photon.
    \item MCNP6 also uses a table based scheme from their condensed history method.
  \end{itemize}
  \begin{equation*}
    p(\mu)d\mu = \frac{(1-\beta^2)}{2(1-\beta\mu)^2}d\mu
  \end{equation*}
  \end{block}  

\end{frame}

%%----------------------------------------------------------------------------%%
\begin{frame}{Known Issues}
  \begin{block}{Absorption at low energies}
  \begin{itemize}
    \item At energies near the cutoff (10 eV) the reaction cross section is dominated by elastic scattering (by order $10^7$ for H)
    \item It is unlikely the electron will scatter below the cutoff energy 
    \item A temporary fix is to raise the cutoff energy (to 15eV for H) to prevent indefinite elastic scattering
    \item MCNP notes this problem and suggests a minimum cutoff energy of 20eV
  \end{itemize}
\end{block}

\end{frame}

%%----------------------------------------------------------------------------%%
\begin{frame}{Testing}
  \begin{block}{Test Problem}
  \begin{itemize}
    \item 10 keV electron delta source in cold Hydrogen
    \item Set surface tallies on 5 spheres of increasing radius to measure the current and flux
    \item Radii of 0.005, 0.001, 0.0015, 0.002 and 0.0025 cm
    \item The electron energy cutoff was set to 15eV and secondary photons where not tracked
  \end{itemize}
\end{block}

  \begin{block}{Verification}
  \begin{itemize}
    \item Test results were verified with MCNP6
    \item The ratio of the  surface flux from MCNP6 and FRENSIE were plotted
    \item The $3\sigma$ rule was used to look at the standard deviation of the flux ratio from the expected value of $1$
    \item 68.27\%, 95.45\% and 99.73\% of the ratios should be within $1, 2 \text{ and } 3 \sigma$ respectfully to verify the sampling routines are the same to a near certainty
  \end{itemize}
\end{block}

\end{frame}

  %%----------------------------------------------------------------------------%%
\begin{frame}{$3~\sigma$ Results}

\begin{table}
\begin{tabular}{l | c | c | c | c | c}
Radius (cm) & 0.0005 & 0.0010 & 0.0015 & 0.0020 & 0.0025 \\
\hline \hline
$1 ~\sigma$ & 72.43\% & 66.84\% & 62.18\% & 57.58\% & 68.00\% \\ 
$2 ~\sigma$ & 96.76\% & 93.16\% & 91.71\% & 93.43\% & 96.00\% \\ 
$3 ~\sigma$ & 99.46\% & 97.89\% & 97.93\% & 98.48\% & 98.50\%
\end{tabular}
\caption{\centering Surface flux ratio and the percentage of energy bins within 1, 2 and 3 $\sigma$ of the expected value}
\end{table}
    
\end{frame}

%%----------------------------------------------------------------------------%%
\begin{frame}{Results}

  \begin{figure}
     \centering
     \includegraphics[width = 0.9\textwidth]{./Sphere1.png}
  \end{figure}


\end{frame}

%%----------------------------------------------------------------------------%%
\begin{frame}{Results}

  \begin{figure}
     \centering
     \includegraphics[width = 0.9\textwidth]{./Sphere2.png}
  \end{figure}


\end{frame}

%%----------------------------------------------------------------------------%%
\begin{frame}{Results}

  \begin{figure}
     \centering
     \includegraphics[width = 0.9\textwidth]{./Sphere3.png}
  \end{figure}


\end{frame}

%%----------------------------------------------------------------------------%%
\begin{frame}{Results}

  \begin{figure}
     \centering
     \includegraphics[width = 0.9\textwidth]{./Sphere4.png}
  \end{figure}


\end{frame}

%%----------------------------------------------------------------------------%%
\begin{frame}{Results}

  \begin{figure}
     \centering
     \includegraphics[width = 0.9\textwidth]{./Sphere5.png}
  \end{figure}


\end{frame}


%%----------------------------------------------------------------------------%%
\begin{frame}{Possible Adjoint Methods}
  \begin{block}{Hybrid Multigroup/Continuous-Energy BFP}
 
    \begin{itemize}
      \item The same basic multigroup cross-section data can be used for forward and adjoint calculations. 
       
      \item The adjoint transport model is nearly identical to the forward making implementation easy 

      \item The transport equation is generalized for Monte Carlo transport of neutral and charged particles.\\ They implement for electrons and photons. 

    \end{itemize}
  \end{block}

\end{frame}

%%----------------------------------------------------------------------------%%
\begin{frame}{Other Possible Adjoint Methods}

     \begin{itemize}
       \item 1980 - Adjoint Electron Transport in the CSDA
           \begin{itemize}
             \item Goudsmit and Saunderson Scattering
           \end{itemize}
                 
       \item 1995 - Adjoint Elecrton-Photon Tansport using BFS in ITS
           \begin{itemize}
             \item Multigroup/Continuous Energy
           \end{itemize}
           
       \item 1996 - Adjoint Multigroup/Continuous Energy BFP Equation

      \item 2005 - Generalized Particle for Couple Adjoint $\gamma$ - $e^-$ - $e^+$ Transport
           \begin{itemize}
             \item CSDA using Moli\`{e}re's multiple scattering
           \end{itemize}
       
     \end{itemize}

\end{frame}


\end{document}
