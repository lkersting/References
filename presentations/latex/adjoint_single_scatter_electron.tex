\documentclass{beamer}
\usetheme[white]{Wisconsin}
\usepackage{longtable}
\usepackage{graphicx}
\usepackage{listings}
\usepackage{color}
%% The amssymb package provides various useful mathematical symbols
\usepackage{amssymb}
%% The amsthm package provides extended theorem environments
\usepackage{amsthm} \usepackage{amsmath}
\usepackage{eqnarray}
\usepackage[mathcal]{euscript} \usepackage{color}
\usepackage{textcomp}
\usepackage{algorithm,algorithmic}
\usepackage[retainorgcmds]{IEEEtrantools}
\usepackage[absolute,overlay]{textpos}
\setlength{\TPHorizModule}{1mm}
\setlength{\TPVertModule}{1mm}
\definecolor{listinggray}{gray}{0.9}
\definecolor{lbcolor}{rgb}{0.9,0.9,0.9}
\lstset{
backgroundcolor=\color{lbcolor},
tabsize=4,
rulecolor=,
language=c++,
basicstyle=\scriptsize,
upquote=true,
aboveskip={1.5\baselineskip},
columns=fixed,
showstringspaces=false,
extendedchars=true,
breaklines=true,
prebreak =
\raisebox{0ex}[0ex][0ex]{\ensuremath{\hookleftarrow}},
frame=single,
showtabs=false,
showspaces=false,
showstringspaces=false,
identifierstyle=\ttfamily,
keywordstyle=\color[rgb]{0,0,1},
commentstyle=\color[rgb]{0.133,0.545,0.133},
stringstyle=\color[rgb]{0.627,0.126,0.941},
}
%% colors
\setbeamercolor{boxheadcolor}{fg=white,bg=red}
\setbeamercolor{boxbodycolor}{fg=black,bg=white}

%%----------------------------------------------------------------------------%%
\author{Luke J. Kersting
\\ UW - Madison
\\ SNL 1341
}
\date{\today}
\title{Electron Mode in FRENSIE}
\begin{document}
\maketitle

%%----------------------------------------------------------------------------%%
\begin{frame}{Single-Event Adjoint Electron Transport}

\begin{block}{Goals}
	\begin{itemize}
	\item Transform forward electron data to adjoint
	\item Use single-event ACE data in transformation
	\item Output usable adjoint single-event electron cross-sections and PDF's
	\end{itemize}

\end{block}

\end{frame}

%%----------------------------------------------------------------------------%%
\begin{frame}{Forward Transport}
\begin{block}{Time-Independent Boltzmann Equation}
\begin{multline}
  \Omega \nabla \phi(\boldsymbol r,E,\boldsymbol\Omega) + 
  \Sigma_t(\boldsymbol{r},E)\phi(\boldsymbol{r},E,\boldsymbol{\Omega}) = \\
  \int\int\Sigma_t(\boldsymbol{r},E')C(\boldsymbol{r},E'\rightarrow E,\boldsymbol{\Omega'}\rightarrow\boldsymbol{\Omega})\phi(\boldsymbol{r},E',\boldsymbol{\Omega'})dE'd\Omega' + 
  S(\boldsymbol{r},E,\boldsymbol{\Omega})
\end{multline}
\end{block}

\begin{block}{Collision Kernel}
\begin{multline}
C(\boldsymbol{r},E'\rightarrow E,\boldsymbol{\Omega'}\rightarrow\boldsymbol{\Omega}) = \\
\sum_A p_A(\boldsymbol{r},E') \sum_j p_{j,A}(E') c_{j,A}(E') f_{j,A}(E'\rightarrow E,\boldsymbol{\Omega'}\rightarrow\boldsymbol{\Omega})
\end{multline}

\begin{equation}
p_A(\boldsymbol{r},E') = \frac{\Sigma_A(\boldsymbol{r},E')}{\Sigma_t(\boldsymbol{r},E')}
~~~~~~~~~~~~~~~~~~
p_{j,A}(E') = \frac{\sigma_{j,A}(E')}{\sigma_A(E')} \nonumber
\end{equation}

\end{block}

\end{frame}

%%----------------------------------------------------------------------------%%
\begin{frame}{Adjoint Transport}

%\begin{block}{Time-Independent Boltzmann Equation}
%\begin{multline}
%  -\Omega \nabla \phi^{\dagger}(\boldsymbol r,E,\boldsymbol\Omega) + 
%  \Sigma_t(\boldsymbol{r},E)\phi^{\dagger}(\boldsymbol{r},E,\boldsymbol{\Omega}) = \\
%  \int\int\Sigma_t(\boldsymbol{r},E)C(\boldsymbol{r},E\rightarrow E',\boldsymbol{\Omega}\rightarrow\boldsymbol{\Omega})\phi^{\dagger}(\boldsymbol{r},E',\boldsymbol{\Omega})dE'd\Omega' + 
%  D(\boldsymbol{r},E,\boldsymbol{\Omega})
%\end{multline}
%\end{block}

\begin{block}{Adjoint Collision Kernel}
\begin{multline}
C^{\dagger}(\boldsymbol{r'},E'\rightarrow E,\boldsymbol{\Omega'}\rightarrow\boldsymbol{\Omega}) = \\
\sum_A p_A^{\dagger}(\boldsymbol{r'},E') \sum_j p^{\dagger}_{j,A}(E') \frac{\sigma_{j,A}(E)c_{j,A}(E)f_{j,A}(E\rightarrow E',\boldsymbol{\Omega}\rightarrow\boldsymbol{\Omega'})}{\sigma^{\dagger}_{j,A}(E')}
\end{multline}

\begin{equation}
\sigma^{\dagger}_{j,A}(E') = \int\sigma_{j,A}(E)c_{j,A}(E)f_{j,A}(E\rightarrow E')dE
\end{equation}

\begin{equation}
p^{\dagger}_A(\boldsymbol{r'},E') = \frac{\Sigma^{\dagger}_A(\boldsymbol{r'},E')}{\Sigma{\dagger}(\boldsymbol{r'},E')}
~~~~~~~~~~~~~~~~~~
p^{\dagger}_{j,A}(E') = \frac{\sigma^{\dagger}_{j,A}(E')}{\sigma^{\dagger}_A(E')} \nonumber
\end{equation}

\end{block}

\end{frame}

%%----------------------------------------------------------------------------%%
\begin{frame}{Adjoint Transport}

First the type of nuclide that the electron interacts with is sampled from:
\begin{equation}
p^{\dagger}_A(\boldsymbol{r'},E') = \frac{\Sigma^{\dagger}_A(\boldsymbol{r'},E')}{\Sigma{\dagger}(\boldsymbol{r'},E')}
\end{equation}

Then the reaction type is sampled from:
\begin{equation}
p^{\dagger}_{j,A}(E') = \frac{\sigma^{\dagger}_{j,A}(E')}{\sigma^{\dagger}_A(E')} \nonumber
\end{equation}

Finally, $E$ and $\Omega$ are sampled from:
\begin{equation}
f^{\dagger}_{j,A}(E'\rightarrow E,\boldsymbol{\Omega'}\rightarrow\boldsymbol{\Omega}) = 
\frac{\sigma_{j,A}(E)c_{j,A}(E)f_{j,A}(E\rightarrow E',\boldsymbol{\Omega}\rightarrow\boldsymbol{\Omega'})}{\sigma^{\dagger}_{j,A}(E')} \nonumber
\end{equation}

\end{frame}


%%----------------------------------------------------------------------------%%
\begin{frame}{Adjoint Cross-Sections}
Electrons reactions are not specifically dependent on the incoming and outgoing angle, 
but instead on $\mu$. Therefore the equations reduced to:
\begin{equation}
\sigma^{\dagger}(E') = \int\int\sigma(E)c(E)f(E \rightarrow E', \mu)dEd\mu
\end{equation}

\begin{equation}
f^{\dagger}(E' \rightarrow E, \mu) = \frac{\sigma(E)c(E)f(E \rightarrow E', \mu)}{\sigma^{\dagger}(E')}
\end{equation}

\begin{equation}
\sigma^{\dagger}(E' \rightarrow E, \mu) = \sigma(E \rightarrow E', \mu)
\end{equation}

\end{frame}

%%----------------------------------------------------------------------------%%
\begin{frame}{Elastic Scattering}
	\begin{itemize}
	\item Their is no energy loss ($ E = E'$)
	\item Adjoint and Forward transport will be exactly the same
	\end{itemize}
\begin{align}
\sigma^{\dagger}(E' \rightarrow E, \mu) = &\sigma(E \rightarrow E', \mu) = \nonumber \\
\sigma^{\dagger}(E, \mu) = &\sigma(E, \mu)
\end{align}

Therefore equations $(6)$ and $(7)$ reduce to:
\begin{equation}
\sigma^{\dagger}(E') = \sigma(E)
\end{equation}

\begin{equation}
f^{\dagger}(E, \mu) = f(E, \mu)
\end{equation}

\end{frame}

%%----------------------------------------------------------------------------%%
\begin{frame}{Elastic Scattering}

\begin{block}{Implementation in FRENSIE}
	\begin{itemize}
	\item Add the ability to take an adjoint particle to the forward class
	\item Add a scatter electron function       
	\end{itemize}
\end{block}

\end{frame}

%%----------------------------------------------------------------------------%%
\begin{frame}{Atomic Excitation}
	\begin{itemize}
	\item Their is no angular deflection
	\item Cross-sections are independent of angle
	\item Each incoming energy will scatter into a unique outgoing energy
	\item There is a one-to-one correspondence between the incoming and outgoing energy
	\end{itemize}

\begin{align}
\sigma^{\dagger}(E' \rightarrow E, \mu) = &\sigma(E \rightarrow E', \mu) = \nonumber \\
\sigma^{\dagger}(E') = &\sigma(E)
\end{align}

\begin{equation}
f^{\dagger}(E', \mu) = \frac{\sigma(E)f(E, \mu)}{\sigma^{\dagger}(E')} = f(E, \mu)
\end{equation}

\end{frame}

%%----------------------------------------------------------------------------%%
\begin{frame}{Atomic Excitation}

\begin{block}{Implementation in FRENSIE}
	\begin{itemize}
	\item Create energy dependent electron energy gain data tables
		\begin{itemize}
		\item Replace the incoming energy with the outgoing energy in the ACE energy loss tables
		\item Use interpolation to create a more uniform energy bin structure
		\end{itemize}
	
	\item Create Adjoint Atomic Excitation class similar to forward case
	\end{itemize}
\end{block}

\end{frame}

%%----------------------------------------------------------------------------%%
\begin{frame}{Bremsstrahlung}
	\begin{itemize}
	\item Angular deflection is assumed to be negligible
	\item Cross-sections are independent of angle
	\end{itemize}

\begin{align}
\sigma^{\dagger}(E' \rightarrow E, \mu) = & \sigma(E \rightarrow E', \mu) =\nonumber \\
\sigma^{\dagger}(E' \rightarrow E) = & \sigma(E \rightarrow E') = \sigma(E)f(E \rightarrow E')
\end{align}

\begin{equation}
\sigma^{\dagger}(E') = \int\sigma(E)f(E \rightarrow E')dE
\end{equation}

\begin{equation}
f^{\dagger}(E' \rightarrow E) = \frac{\sigma(E)f(E \rightarrow E')}{\sigma^{\dagger}(E')}
\end{equation}


\end{frame}

%%----------------------------------------------------------------------------%%
\begin{frame}{Bremsstrahlung}

\begin{block}{Implementation in FRENSIE}
	\begin{itemize}
	\item Create 2D electron energy gain pdf data tables
		\begin{itemize}
		\item Numerically integrate the cross section for a given energy loss over all incident energies
		\item Use interpolation to create a energy bin structure
		\item Create a pdf for each outgoing energy bin
		\end{itemize}
	
	\item Create new adjoint Bremsstrahlung class
	\end{itemize}
\end{block}

\end{frame}

%%----------------------------------------------------------------------------%%
\begin{frame}{Electroionization}
	\begin{itemize}
	\item A second electron is produced
	\item There is a unique angle for each $E \rightarrow E'$ pair
	\end{itemize}

\begin{equation}
p(E \rightarrow E', \mu) = f(E \rightarrow E')
\end{equation}

\begin{equation}
\sigma^{\dagger}(E') = \int\sigma(E)f(E \rightarrow E')dE
\end{equation}

\begin{equation}
f^{\dagger}(E'\rightarrow E, \mu) = \frac{\sigma(E)f(E \rightarrow E')}{\sigma^{\dagger}(E')} 
\end{equation}

\end{frame}

%%----------------------------------------------------------------------------%%
\begin{frame}{Braking Down the Adjoint Electroionization Cross-Section}
	\begin{itemize}
	\item The adjoint cross-section can be broken into two parts corresponding to the primary and secondary particle
	\item The primary particle has a energy range: ~~~~$E/2 \leq E' \leq E$
	\item The secondary particle has a energy range: ~$E_{min} \leq E' \leq E/2$
	\end{itemize}


\begin{align}
\sigma^{\dagger}(E') = &\sigma^{\dagger}_{prim}(E') + \sigma^{\dagger}_{sec}(E') \nonumber \\
= & \int\sigma(E) \Big( f_{prim}(E \rightarrow E') + f_{sec}(E \rightarrow E') \Big)dE
\end{align}

\end{frame}

%%----------------------------------------------------------------------------%%
\begin{frame}{Electroionization}

\begin{block}{Implementation in FRENSIE}
	\begin{itemize}
	\item Create 2D electron energy gain pdf data tables
		\begin{itemize}
		\item Numerically integrate the cross section for a given outgoing electron energy over all incident electron energies
		\item Use interpolation to create a energy bin structure
		\item Create a pdf for each knock-on energy bin
		\end{itemize}
	
	\item Create new adjoint Electroionization template class
	\end{itemize}
\end{block}

\end{frame}

%%----------------------------------------------------------------------------%%
\end{document}
