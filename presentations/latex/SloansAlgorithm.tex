\documentclass{beamer}
\usetheme[white]{Wisconsin}
\usepackage{longtable}
\usepackage{graphicx}
\usepackage{listings}
\usepackage{color}
%% The amssymb package provides various useful mathematical symbols
\usepackage{amssymb}
%% The amsthm package provides extended theorem environments
\usepackage{amsthm} \usepackage{amsmath}
%% \usepackage{eqnarray}
\usepackage[mathcal]{euscript} \usepackage{color}
\usepackage{textcomp}
\usepackage{algorithm,algorithmic}
\usepackage[retainorgcmds]{IEEEtrantools}
\usepackage[absolute,overlay]{textpos}
  \setlength{\TPHorizModule}{1mm}
  \setlength{\TPVertModule}{1mm}
\definecolor{listinggray}{gray}{0.9}
\definecolor{lbcolor}{rgb}{0.9,0.9,0.9}
\lstset{
  backgroundcolor=\color{lbcolor},
  tabsize=4,
  rulecolor=,
  language=c++,
  basicstyle=\scriptsize,
  upquote=true,
  aboveskip={1.5\baselineskip},
  columns=fixed,
  showstringspaces=false,
  extendedchars=true,
  breaklines=true,
  prebreak =
  \raisebox{0ex}[0ex][0ex]{\ensuremath{\hookleftarrow}},
  frame=single,
  showtabs=false,
  showspaces=false,
  showstringspaces=false,
  identifierstyle=\ttfamily,
  keywordstyle=\color[rgb]{0,0,1},
  commentstyle=\color[rgb]{0.133,0.545,0.133},
  stringstyle=\color[rgb]{0.627,0.126,0.941},
}

%% colors
\setbeamercolor{boxheadcolor}{fg=white,bg=UWRed}
\setbeamercolor{boxbodycolor}{fg=black,bg=white}

%%----------------------------------------------------------------------------%%
\author{Luke J. Kersting
    \\ NEEP
    \\ University of Wisconsin - Madison
    \\ SNL
}

\date{\today}
\title{Calculating Moments Using Sloan's Algorithm}
\begin{document}
\maketitle


%%----------------------------------------------------------------------------%%
\begin{frame}{Moments}

  \begin{block}{Legendre}
  \begin{equation*}
    f_l = \int_{-1}^{1}f(\mu)P_l(\mu)d\mu
  \end{equation*}
  \end{block}
    
  \begin{block}{Gauss}
  \begin{equation*}
    M_n = \int_{-1}^{1}\mu^nf(\mu)d\mu = \sum_{l=0}^{\inf}\frac{2l+1}{2}
    f_l\int_{-1}^{1}\mu^nP_l(\mu)d\mu
  \end{equation*}
  \end{block}

  \begin{block}{Radau}
  \begin{equation*}
    M^*_n = M_n - M_{n+1}
  \end{equation*}
  \end{block}
    

\end{frame}


%%----------------------------------------------------------------------------%%
\begin{frame}{Negative Weights}

\begin{block}{Variance}
  If the Variance is negative a negative weight will be preoduced
  \begin{equation*}
    \sigma^2_i = N_i/N_{i-1}
  \end{equation*}
  Therefore the variance is negative if the normalization factor $N$ is negative. \\
\end{block}

\begin{block}{Normalization Factor}
  \begin{equation*}
    N_i = \sum_{k=0}^{i} A_{i,k}M^*_{k+i}
  \end{equation*}
  Where $A_{i,k}$ are the coefficients of the orthogonal polynomial $Q_n$
\end{block}

\end{frame}

%%----------------------------------------------------------------------------%%
\begin{frame}{The n=2 case}

The abscissa for $x_n = 1$. Sloan shows (B-57) that the weight of the last 
abscissa is:

$$w_n = 1 - \sum_{i=1}^{n-1}\frac{N_{i-1}}{Q_{i-1}(1)Q_i(1)}$$

Plugging in the recursion relationship for $Q_i(x)$ (B-117) \\
and requiring $w_n>0$ and solving for the mean coefficient,$\mu_{n-1}$ we get:

$$\mu^{max}_{n-1} \leq 1 - \sigma^2_{n-2}\frac{Q_{n-3}(1)}{Q_{n-2}(1)} - G_{n-2}(1)$$

Where (B-143b):

$$G_{n-2}(1) = \frac{N_{n-2}}{[Q_{n-2}(1)]^2}\left[ 1 -\sum_{k=1}^{n-2} 
                \frac{N_{k-1}}{Q_{k-1}(1)Q_{k}(1)}\right]^{-1}$$


\end{frame}

%%----------------------------------------------------------------------------%%
\begin{frame}{The n=2 case continued}

For the $n=2$ case the reduces to:

$$\mu^{max}_1 \leq 1 - \sigma^2_{0}\frac{0}{Q_{0}(1)} - G_{0}(1)$$

Where:

$$G_{0}(1) = \frac{N_{0}}{[Q_{0}(1)]^2} = N_{0} = M^*_0 = f_0 - f_1 = 1 - f_1$$

Therefore sloan requires:

$$\mu_1 \leq f_1 \rightarrow \frac{M^*_1}{M^*_0}\leq f_1$$

$$\frac{-\frac{1}{3} + f_1 - \frac{2}{3} f_2}{1 - f_1}\leq f_1$$

$$2f_2 + 1\geq 3f_1^2$$


\end{frame}

\end{document}
