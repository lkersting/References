\documentclass{beamer}
\usetheme[white]{Wisconsin}
\usepackage{longtable}
\usepackage{listings}
\usepackage{color}
%% The amssymb package provides various useful mathematical symbols
\usepackage{amssymb}
%% The amsthm package provides extended theorem environments
\usepackage{amsthm} \usepackage{amsmath}
%% \usepackage{eqnarray}
\usepackage[mathcal]{euscript} \usepackage{color}
\usepackage{textcomp}
\usepackage{algorithm,algorithmic}
\usepackage[retainorgcmds]{IEEEtrantools}
\usepackage[absolute,overlay]{textpos}
  \setlength{\TPHorizModule}{1mm}
  \setlength{\TPVertModule}{1mm}
\definecolor{listinggray}{gray}{0.9}
\definecolor{lbcolor}{rgb}{0.9,0.9,0.9}
\lstset{
  backgroundcolor=\color{lbcolor},
  tabsize=4,
  rulecolor=,
  language=c++,
  basicstyle=\scriptsize,
  upquote=true,
  aboveskip={1.5\baselineskip},
  columns=fixed,
  showstringspaces=false,
  extendedchars=true,
  breaklines=true,
  prebreak =
  \raisebox{0ex}[0ex][0ex]{\ensuremath{\hookleftarrow}},
  frame=single,
  showtabs=false,
  showspaces=false,
  showstringspaces=false,
  identifierstyle=\ttfamily,
  keywordstyle=\color[rgb]{0,0,1},
  commentstyle=\color[rgb]{0.133,0.545,0.133},
  stringstyle=\color[rgb]{0.627,0.126,0.941},
}

%% colors
\setbeamercolor{boxheadcolor}{fg=white,bg=UWRed}
\setbeamercolor{boxbodycolor}{fg=black,bg=white}

%%----------------------------------------------------------------------------%%
\author{Luke J. Kersting
    \\ NEEP
    \\ University of Wisconsin - Madison
    \\ FRENSIE Meeting
}

\date{\today}
\title{Update: Electron Mode in FRENSIE}
\begin{document}
\maketitle

%%----------------------------------------------------------------------------%%
\begin{frame}{Outline}

  \begin{block}{Electron Mode}
    \begin{itemize}
      \item Single Scattering Events from 100 GeV to 10 eV
      \item Elastic, Bremsstrahlung, Electroionization, Atomic Excitation 
      \item Secondary particles created, but photons not tracked
      \item Atomic relaxation implemented
    \end{itemize}
  \end{block}
    
  \begin{block}{Adjoint Papers}
    \begin{itemize}
      \item Hybrid Multigroup/Continuous-Energy Monte Carlo using Boltzmann-Fokker-Planck Equation
      \item Discrete Scattering Angles and Discrete Energy Losses
    \end{itemize}    
  \end{block}  

\end{frame}

%%----------------------------------------------------------------------------%%
\begin{frame}{Electron Mode}

  \begin{block}{Capabilities}
    \begin{itemize}
      \item Single Scattering Events from 100 GeV to 10 eV
      \item Elastic, Bremsstrahlung, Electroionization, Atomic Excitation 
      \item Secondary particles created, but photons not tracked
      \item Atomic relaxation implemented
    \end{itemize}
  \end{block}
    
  \begin{block}{Problems}
    \begin{itemize}
      \item Absorption at low energies
      \item Negative energy from Electroionization
    \end{itemize}    
  \end{block}  

\end{frame}

%%----------------------------------------------------------------------------%%
\begin{frame}{Absorption at low energies}

  \begin{itemize}
    \item At energies near the cutoff (10 eV) the reaction cross section is dominated by elastic scattering (by order $10^7$ for H)
    \item It is unlikely the electron will scatter below the cutoff energy 
    \item A temporary fix is to raise the cutoff energy (to 15eV for H) to prevent indefinite elastic scattering
    \item MCNP notes this problem and suggests a minimum cutoff energy of 20eV
  \end{itemize}

\end{frame}

%%----------------------------------------------------------------------------%%
\begin{frame}{Negative energy from Electroionization}
  
  \begin{itemize}
    \item ACE tables provide CDF of the knock-on energy, $E_{knock}$, based on the incident electron energy.
    \item When the incident electron energy is between two tables a weighted random variable is used to chose the appropriate table
    \item This can result in a $E_{knock}$ that is larger than physically possible
     \item In this case the energy of incident electron is reduce to $1E{-15}$
    \item This can be avoided by interpolation between tables, which is more computationally expensive
     \item MCNP does something similar
  \end{itemize}


\end{frame}

%%----------------------------------------------------------------------------%%
\begin{frame}{Differences in Results}

    \begin{itemize}
     \item Had discrepencies from MCNP caused by knock-on electron ejection angles
 


  \end{itemize}


\end{frame}
  


\end{document}
