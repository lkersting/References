\documentclass{beamer}
\usetheme[white]{Wisconsin}
\usepackage{longtable}
\usepackage{graphicx}
\usepackage{listings}
\usepackage{color}
%% The amssymb package provides various useful mathematical symbols
\usepackage{amssymb}
%% The amsthm package provides extended theorem environments
\usepackage{amsthm} \usepackage{amsmath}
%% \usepackage{eqnarray}
\usepackage[mathcal]{euscript} \usepackage{color}
\usepackage{textcomp}
\usepackage{algorithm,algorithmic}
\usepackage[retainorgcmds]{IEEEtrantools}
\usepackage[absolute,overlay]{textpos}
  \setlength{\TPHorizModule}{1mm}
  \setlength{\TPVertModule}{1mm}
\definecolor{listinggray}{gray}{0.9}
\definecolor{lbcolor}{rgb}{0.9,0.9,0.9}
\lstset{
  backgroundcolor=\color{lbcolor},
  tabsize=4,
  rulecolor=,
  language=c++,
  basicstyle=\scriptsize,
  upquote=true,
  aboveskip={1.5\baselineskip},
  columns=fixed,
  showstringspaces=false,
  extendedchars=true,
  breaklines=true,
  prebreak =
  \raisebox{0ex}[0ex][0ex]{\ensuremath{\hookleftarrow}},
  frame=single,
  showtabs=false,
  showspaces=false,
  showstringspaces=false,
  identifierstyle=\ttfamily,
  keywordstyle=\color[rgb]{0,0,1},
  commentstyle=\color[rgb]{0.133,0.545,0.133},
  stringstyle=\color[rgb]{0.627,0.126,0.941},
}

%% colors
\setbeamercolor{boxheadcolor}{fg=white,bg=UWRed}
\setbeamercolor{boxbodycolor}{fg=black,bg=white}

%%----------------------------------------------------------------------------%%
\author{Luke J. Kersting
    \\ NEEP
    \\ University of Wisconsin - Madison
    \\ FRENSIE Meeting
}

\date{\today}
\title{Review of Summer at Sandia National Lab}
\begin{document}
\maketitle

%%----------------------------------------------------------------------------%%
\begin{frame}{Outline}

  \begin{itemize}
    \item Interaction at Sandia
    \item Condensed History's History
    \item Moment Preserving Method
    \item The Evaluated Electron Data Library (EEDL)
    \item Adjoint Analog Transport
  \end{itemize}

\end{frame}

  %%----------------------------------------------------------------------------%%
  \begin{frame}{Condensed History Electron Transport}

  \begin{block}{Electron’s charge complicates analog simulation of events}
    \begin{itemize}
      \item Neutral particles may scatter a couple dozen times over a distance
      \item SElectrons may scatter 10,000 or more times over the same distance
      \item Analog simulation is impractical
      \item Simulation of hard electron transport events
    \end{itemize}
  \end{block}
    
  \begin{block}{A “Condensed” Random Walk method was developed}
    \begin{itemize}
      \item Electrons are moved a set step length
      \item A multiply scattering theory is sampled to find the outgoing direction
      \item The Continuous Slowing Down Approximation (CSDA) is used to calculate energy loss
      \item Production of secondary are averaged
      \item Approximations don’t hold below 1 keV
      \item Uses infinite medium approximation
    \end{itemize}    
  \end{block}  

\end{frame}

\end{document}